%  article.tex (Version 3.3, released 19 January 2008)
%  Article to demonstrate format for SPIE Proceedings
%  Special instructions are included in this file after the
%  symbol %>>>>
%  Numerous commands are commented out, but included to show how
%  to effect various options, e.g., to print page numbers, etc.
%  This LaTeX source file is composed for LaTeX2e.

%  The following commands have been added in the SPIE class 
%  file (spie.cls) and will not be understood in other classes:
%  \supit{}, \authorinfo{}, \skiplinehalf, \keywords{}
%  The bibliography style file is called spiebib.bst, 
%  which replaces the standard style unstr.bst.  

\documentclass[]{spie}  %>>> use for US letter paper
%%\documentclass[a4paper]{spie}  %>>> use this instead for A4 paper
%%\documentclass[nocompress]{spie}  %>>> to avoid compression of citations
%% \addtolength{\voffset}{9mm}   %>>> moves text field down
%% \renewcommand{\baselinestretch}{1.65}   %>>> 1.65 for double spacing, 1.25 for 1.5 spacing 
%  The following command loads a graphics package to include images 
%  in the document. It may be necessary to specify a DVI driver option,
%  e.g., [dvips], but that may be inappropriate for some LaTeX 
%  installations. 
\usepackage[]{graphicx}
% **************** Javier Added **********************
\usepackage{tikz} % for flow charts
\usetikzlibrary{shapes,arrows,positioning,shadows,calc,plotmarks}
% The data files, written on the first run.

\usepackage[textwidth=3.7cm]{todonotes} 
\usepackage{hyperref}
\hypersetup{
    % bookmarks=true,         % show bookmarks bar?
    unicode=false,          % non-Latin characters in AcrobatÕs bookmarks
    pdftoolbar=true,        % show AcrobatÕs toolbar?
    pdfmenubar=true,        % show AcrobatÕs menu?
    pdffitwindow=false,     % window fit to page when opened
    pdfstartview={FitH},    % fits the width of the page to the window
    pdftitle={L8's potential for water constituents retrieval },    % title
    pdfauthor={Javier Concha},     % author
    pdfsubject={Subject},   % subject of the document
    pdfcreator={Creator},   % creator of the document
    pdfproducer={Producer}, % producer of the document
    pdfkeywords={keyword1} {key2} {key3}, % list of keywords
    pdfnewwindow=true,      % links in new window
    colorlinks=true,       % false: boxed links; true: colored links
    linkcolor=blue,          % color of internal links
    citecolor=cyan,        % color of links to bibliography
    filecolor=magenta,      % color of file links
    urlcolor=green           % color of external links
}

\usepackage[all]{hypcap} % to see figure with hyper ref

\usepackage{filecontents}% http://ctan.org/pkg/filecontents
\usepackage{silence}% http://ctan.org/pkg/silence
\WarningFilter{latex}{Overwriting file}% Remove LaTeX warnings starting with "Overwriting file"
\begin{filecontents*}{linereg.data}
#x y
0 4
10 24
\end{filecontents*} 

\begin{filecontents*}{linereg2.data}
#x y
2 8
8 20
\end{filecontents*} 
% ****************************************************

\title{A Model-Based ELM for Atmospheric Correction over Case 2 Water with Landsat 8} 

%>>>> The author is responsible for formatting the 
%  author list and their institutions.  Use  \skiplinehalf 
%  to separate author list from addresses and between each address.
%  The correspondence between each author and his/her address
%  can be indicated with a superscript in italics, 
%  which is easily obtained with \supit{}.

\author{Javier A. Concha and John R. Schott
\skiplinehalf
Digital Imaging and Remote Sensig Lab\\Chester F. Carlson Center for Imaging Science\\Rochester Institute of Technology\\ 54 Lomb Memorial Dr., Rochester, NY 14623, USA\\
}

%>>>> Further information about the authors, other than their 
%  institution and addresses, should be included as a footnote, 
%  which is facilitated by the \authorinfo{} command.

\authorinfo{Further author information: (Send correspondence to Javier A. Concha)\\ E-mail: jxc4005@rit.edu, Telephone: 1-(585) 290-3145}
%%>>>> when using amstex, you need to use @@ instead of @
 

%%%%%%%%%%%%%%%%%%%%%%%%%%%%%%%%%%%%%%%%%%%%%%%%%%%%%%%%%%%%% 
%>>>> uncomment following for page numbers
% \pagestyle{plain}    
%>>>> uncomment following to start page numbering at 301 
%\setcounter{page}{301} 
 
  \begin{document} 
  \maketitle 

%%%%%%%%%%%%%%%%%%%%%%%%%%%%%%%%%%%%%%%%%%%%%%%%%%%%%%%%%%%%% 
\begin{abstract}
The Landsat 8 satellite, recently launched (February 2013), carries the next generation of Landsat sensors and extends over 40 years of continuous imaging acquisition. Landsat 8, with its improved spectral coverage and radiometric resolution, has the potential to dramatically improve our ability to simultaneously retrieve the three primary Color Producing Agents (CPAs) (chlorophyll, colored dissolved organic matter, and suspended minerals) from water bodies and considering its 30-meter resolution should be especially useful for studying the nearshore environment.

In the Case 2 water problem, accurate atmospheric correction is essential, yet remains a significant source of water-constituent retrieval error particularly since the sensor-reaching signal, due to water, is very small compared to the signal from atmospheric effects. Furthermore, the standard black target assumption commonly used for open ocean studies is not valid in turbid water due to the presence of water-leaving signal in the near infrared (NIR). In this work, a modified version of the traditional empirical line method (ELM) has been developed, which utilizes reflectance from both an in-water radiative transfer model (Hydrolight) and a reflectance product (Landsat surface reflectance product) to atmospherically correct Landsat 8 images. This method employs pseudo-invariant feature (PIF) pixel extraction to mask urban landscape from the reflectance product for the bright pixel determination. For the dark pixel, Hydrolight is used to obtain the field spectra that replaces ground-truth measurements normally used in the traditional ELM. The radiance values for the dark and bright pixels are extracted from the corresponding regions in the Landsat 8 image. Initial results of this method are compared to results obtained from a traditional ELM for validation.\end{abstract}

%>>>> Include a list of keywords after the abstract 

\keywords{Atmospheric Correction, Case 2 Water, Landsat 8, Inland and Coastal Water, Empirical Line Method, ELM, Empirical Line Fit, ELF, Operational Land Imager, OLI}
%%%%%%%%%%%%%%%%%%%%%%%%%%%%%%%%%%%%%%%%%%%%%%%%%%
%%%%%%%%%%%%%%%%%%%%%%%%%%%%%%%%%%%%%%%%%%%%%%%%%%

\section{Introduction} 
The objective in this research is to identify a suitable approach to atmospherically correct the type of dataset provided by the Operation Land Imager (OLI) instrument onboard of the Landsat 8 satellite for studies over Case 2 waters (coastal and inland waters). The OLI instrument is a multispectral sensor with four bands in the visible (coastal, blue, green and red bands), one in the near infrared (NIR) and two in the short wave infrared (SWIR) plus a panchromatic band. It has a 30-meter pixel spot size, 12 bit quantization and an improved signal-to-noise ratio when compared with its predecessors. These characteristics raise the potential to use the Landsat 8 satellite for the monitoring of coastal and inland waters, filling a gap for products derived from satellite imagery over water bodies that cannot be resolvable by low resolution satellites such as the Sea-viewing Wide Field-of-view Sensor (SeaWiFS) and the Moderate Resolution Imaging Spectroradiometer (MODIS) instruments. A portion of a Landsat 8 image showing part of Lake Ontario, some nearby ponds and Downtown Rochester, NY is shown in \autoref{fig:Scene} as an example of water bodies that cannot be resolvable using a 1000-meter pixel resolution satellite (e.g. SeaWiFS or MODIS). The atmospheric correction method developed in this paper is applied over this Landsat 8 image.

\begin{figure}[htb]
  	\centering
  	\includegraphics[height=7cm]{/Users/javier/Desktop/Javier/PHD_RIT/Latex/Proposal/Images/LC80160302013262LGN00subset.jpg}
  \caption{Portion of the Landsat 8 image to be corrected showing part of the Lake Ontario, nearby ponds and Downtown Rochester. \label{fig:Scene} } 
\end{figure}

The atmospheric correction of satellite imagery is a complex task to perform over water because the signal leaving the water that reaches the sensor is very small compared to the signal reaching the sensor from atmospheric scattering. Most of the atmospheric correction algorithms applied to ocean color satellites (i.e. SeaWiFS and MODIS) are not suitable for turbid coastal waters because the {\it black pixel assumption} (a.k.a. black target assumption) cannot be applied to these types of waters. Different modifications to these algorithms have been suggested~\cite{Patt2003}, but they still could fail over highly turbid waters. This work presents an alternative approach: an in-scene method based on the well-known empirical line method (ELM) (a.k.a. empirical line fit or ELF).
%%%%%%%%%%%%%%%%%%%%%%%%%%%%%%%%%%%%%%%%%%%%%%%%%%
%%%%%%%%%%%%%%%%%%%%%%%%%%%%%%%%%%%%%%%%%%%%%%%%%%
\section{Model-based ELM} 

\subsection{Traditional Empirical Line Method}
The method uses a model-based empirical line method (ELM) based on previous work done for simulated synthetic OLI data\cite{Gerace:2013,Gerace:2012}. While this new method is based on the traditional ELM method, this model-based ELM method tries to avoid the measurement of ground truth at every sensor passover by using pseudo-invariant features (PIF) in the scene as one target along with an estimation of water reflectivity for an open lake region for the other target. The two targets used in this model-based ELM are referred to in this documents as the {\it bright pixel} and the {\it dark pixel}.

The {\it empirical line method} (ELM) is a method for calibration of image data to reflectance that uses ground truth. It assumes there are empirical relationships between radiance and reflectance. These empirical relationships require ground measurement of reflectances at a number of targets\cite{Smith:1999} (at least two). The ELM uses a linear regression in each band to relate digital counts (a.k.a. digital number or DN) or radiance to reflectance\cite{Schott}. In general, the ground truth can be either control panels or ad hoc control surfaces of known reflectance. These ground truth objects need to be approximately Lambertian to minimize any errors that could be introduced by sensor view angle effects. Also, these calibration targets are assumed flat, with no neighboring obscuration, and homogeneous as well. The ELM method generally assumes that the atmosphere is constant over the complete scene. If that is not the case, corrections must be made for changes in the atmosphere over the scene. The regression to be solved for each band in the ELM method is given by
\begin{equation}
	\label{eq:ELM} 
	L = m\times r_d + b
\end{equation}
where $L$ is the radiance reaching the sensor value, $m$ is the slope of the regression, $r_d$ is the reflectance of the target, and $b=L_u$ is the intercept, with $L_u$ the upwelled radiance or path radiance. \autoref{fig:ELM} illustrates this concept. Then, the reflectance of any Lambertian objects can be calculated by rearranging \autoref{eq:ELM}, i.e. $r_d=(L-b)/m$. In order to solve this regression, i.e. determine the value of $m$ and $b$, we need to have at least two targets with known radiance $L$ and reflectance $r_d$. After $m$ and $b$ have been determined, the reflectance of each pixel at each wavelength can be calculated from its radiance value from the image.
\begin{figure}[htb]
	\centering
% \resizebox{9cm}{!}{%
\begin{tikzpicture}[x=4ex,y=1ex]
 	%axis
	\draw (0,0) -- coordinate (x axis mid) (10,0);
    \draw (0,0) -- coordinate (y axis mid) (0,30);
    %ticks
 %  	\foreach \x in {0,...,10}
 %   		\draw (\x,1pt) -- (\x,-3pt)
	% node[anchor=north] {\x};
 %  	\foreach \y in {0,5,...,30}
 %   		\draw (1pt,\y) -- (-3pt,\y) 
 %   			node[anchor=east] {\y}; 
    %labels      
	\node[below=0ex] at (8,0) {\small $Band_i~~reflectance~(r_d)$};
	\node[rotate=90] at (-.5,23) {\small $Band_i~~Radiance~(L)$};

	\node[below=.2ex] at (-2.1,4.5) {\scriptsize $b=$offset};
	\node[below=1.4ex] at (-2.1,4.0) {\scriptsize (path radiance)};
	\draw[rotate=90,|<->|] (0,1) -- coordinate (x axis mid) (1,1);

	\node[below=0ex] at (2,15) {\small Dark Object};
	\draw[arrows=-triangle 45] (2,12.5) -- (2,9);

	\node[below=0ex] at (4,20) {\small $m=$ Slope};
	\draw[arrows=-triangle 45] (4,17.5) -- (5,14.5);

	\node[below=0ex] at (7,27) {\small Bright Object};
	\draw[arrows=-triangle 45] (7,24.5) -- (7.9,20.5);

	\node[below=0ex] at (8,9) {\small $r_d=(L-b)/m$};

	%plots
	\draw plot 
		file {linereg.data};
	\draw plot[mark=*] 
		file {linereg2.data};

\end{tikzpicture}
  % }%close \resizebox
\caption{Regression used in ELM to solve the linear relationship between reflectance $r_d$ and radiance $L$ using a dark and bright pixel from the scene. \label{fig:ELM}}
\end{figure}

%  \autoref{fig:ELM} illustrates this concept
% \begin{figure}[htb]
%   	\centering
%   	\includegraphics[height=7cm]{/Users/javier/Desktop/Javier/PHD_RIT/Latex/Proposal/Images/ELMgraph.pdf}
%   \caption{Regression used in ELM to solve the linear relationship between reflectance $r_d$ and radiance $L$ using a dark and bright pixel from the scene. \label{fig:ELM} } 
% \end{figure}

\subsection{Pseudo-Invariant Feature Extraction}

This model-based ELM method employs a PIF pixel extraction for the bright pixel determination. Pseudo-invariant targets are defined as targets whose reflectivity properties do not change rapidly between different times of collection\cite{Schott:1988}. Examples of pseudo-invariant target are urban features in the scene.  The PIF extraction isolates the pseudo-invariant features from the digital imagery. In our case, the PIF are the man-made urban features in a scene. A flowchart of the process is shown in \autoref{fig:PIFflowchart}. 

\begin{figure}[htb]
	\centering

\resizebox{10cm}{!}{%	
  \begin{tikzpicture}[node distance=5ex, auto]
          \tikzset{
                  basenode/.style={rectangle,rounded corners,draw=black,very thick, inner sep=1em, minimum size=3em, text centered,text width=15ex},
                  productnode/.style={ellipse,rounded corners,draw=black, very thick, text centered,text width=9ex},
                  myarrow/.style={->,>=stealth',thick, double = black},
                  mylabel/.style={text width=8em, text centered}
          }
          % SWIR branch
          \node[basenode] (SWIR) {SWIR 2\\ Band};
          \node[basenode, below=of SWIR] (TS1) {Mask by Threshold (upward)};
          \node[align=left, right=0.0 of TS1] (C1) {Urban\\Veget.\\Water};
          \node[align=left, right=-0.15 of C1] (C2) {ON\\ON\\OFF};

          % Ratio branch
          \node[basenode, right=20ex of SWIR] (Ratio) {Ratio\\ NIR Band/ Red Band};
          \node[basenode, below=of Ratio] (TS2) {Mask by Threshold (downward)};
          \node[align=left, right=0.0 of TS2] (C3) {Urban\\Veget.\\Water};
          \node[align=left, right=-0.15 of C3] (C4) {ON\\OFF\\ON};

          % AND
          \path (TS1.south)--(TS2.south) node[pos=.5,below=10ex] (AND) {.AND.};


          % PIF Mask
          \node[basenode, below=of AND] (PIFMask){PIF Mask};
          \node[align=left, left=0.85 of PIFMask] (C5) {Urban\\Veget.\\Water};
          \node[align=left, right=-0.15 of C5] (C6) {ON\\OFF\\OFF};

          \node[basenode, below=of TS2,right=10ex of AND] (Image) {Image};
          \path (Image.south)--(PIFMask.east) node[below=of Image,right=10ex of PIFMask] (AND2) {.AND.};
          \node[basenode, right=10ex of AND2] (PIFIm){PIF Image};

          \draw[myarrow] (SWIR)--(TS1);
          \draw[myarrow] (Ratio)--(TS2);
          \draw[myarrow] (TS1)--(AND);
          \draw[myarrow] (TS2)--(AND);
          \draw[myarrow] (AND)--(PIFMask);
          \draw[myarrow] (Image)--(AND2);
          \draw[myarrow] (PIFMask)--(AND2);
          \draw[myarrow] (AND2)--(PIFIm);
  \end{tikzpicture}
  }%close \resizebox
\caption{Illustration of the logic used to segment PIF features. \label{fig:PIFflowchart}}
\end{figure}

The PIF extraction from digital imagery proceeds in the following fashion (see \autoref{fig:PIFflowchart}). An infrared-to-red ratio image is very effective in the classification of water, vegetation, and urban features. The vegetation in this ratio image will tend to have a high brightness when compared to the urban features and water brightness. This infrared-to-red ratio image can be derived from the quotient of the NIR band (band 4 for Landsat 5 and band 5 for Landsat 8) and the red band (band 3 for Landsat 5 and band 4 for Landsat 8), as seen in \autoref{fig:PIFflowchart}. This ratio image is thresholded from the high digital count values downward to create a mask so the high brightness pixels (vegetation pixels) are eliminated  from the image, that is, these pixels are set to a value of zero and the rest (water and urban pixels) to a value of one. Then, the SWIR 2 band (band 7 in both Landsat 5 and Landsat 8) is used to eliminate the water pixels and wet soil or wet vegetation from the previous mask since water has nearly zero reflectance in this spectral region, therefore the water pixels will exhibit a low value when compare to the rest of the pixels. This SWIR 2 band is thresholded from the low brightness, upward. A mask is created by assigning a value of zero to the low brightness pixels (water pixels) and a value of one to the rest (urban features and vegetation). Finally, the two masks created are combined using a logical .AND. resulting in a mask that will have a value of one only in the urban feature pixels, i.e. the PIFs, as shown in the diagram of \autoref{fig:PIFflowchart}. This mask will be named PIF mask for the rest of this document. An example of a PIF mask is illustrated in \autoref{fig:PIFmask}. A false color image of Downtown Rochester, NY is shown in \autoref{fig:PIFmask}.(a) (vegetation in red) and a RGB image of the same area with the PIF mask applied is shown in \autoref{fig:PIFmask}.(b) (urban features in bright color while masked pixels in black). 

% \vspace{-.3cm}
\begin{figure}[htb]
  \begin{minipage}[c]{0.48\linewidth}
    \centering
      \includegraphics[trim=30 0 30 0,clip,height=6cm]{/Users/javier/Desktop/Javier/PHD_RIT/Latex/Proposal/Images/DTROCL8falsecolor.jpg}  
    % \vspace{1.5cm}
    \centerline{(a)}\medskip
  \end{minipage}
  \hfill
  \begin{minipage}[d]{0.48\linewidth}
    \centering
      \includegraphics[trim=30 0 30 0,clip,height=6cm]{/Users/javier/Desktop/Javier/PHD_RIT/Latex/Proposal/Images/PIFmaskApplied.jpg}
    % \vspace{1.5cm}
    \centerline{(b)}\medskip
  \end{minipage}
  \caption{PIF mask determination over downtown Rochester, NY. (a) False color image, with vegetation in red and (b) PIF mask applied to the RGB image. \label{fig:PIFmask} } 
\end{figure}

\subsection{Bright Pixel Determination}

As mentioned above, the PIF mask is used to determine the radiance and reflectance values of the bright pixel spectrum from an urban area present in the scene, as it is in the case of the Downtown Rochester scene shown in \autoref{fig:PIFmask} and \autoref{fig:Scene}.  The radiance values are obtained from the Landsat 8 image to be atmospherically corrected, while the reflectance values are obtained from a corresponding image of the Landsat surface reflectance product. The Landsat climate data record (CDR) surface reflectance product is part of the higher-level Landsat data product to support land surface change study developed by USGS \cite{LandsatCDR}. For the Landsat 8 image shown in \autoref{fig:Scene}, a total of nine corresponding Landsat 5 scenes with clear sky conditions were available in the Landsat reflectance product. One PIF mask for each of these nine scenes was created using ENVI\cite{ENVIUserGuide}. In addition, one PIF mask was created from the Landsat 8 radiance image. Finally, these 10 PIF masks were combined using a logical .AND. to create a ``master'' PIF mask. Then, the master PIF mask was applied to the Landsat 8 image and one of the Landsat surface reflectance product, and the statistics of all pixels that passed through this master PIF mask were calculated in the software ENVI. An example of the mean values obtained for the Landsat 8 and Landsat 5 reflectance product are shown \autoref{tab:PIFvalues}. 
% \begin{figure}[!ht]
%   \begin{minipage}[c]{0.48\linewidth}
%     \centering
%       \includegraphics[height=9cm]{/Users/javier/Desktop/Javier/PHD_RIT/Latex/Proposal/Images/PIFstatCDR.png}  
%     % \vspace{1.5cm}
%     \centerline{(a)}\medskip
%   \end{minipage}
%   \hfill
%   \begin{minipage}[d]{0.48\linewidth}
%     \centering
%       \includegraphics[height=9cm]{/Users/javier/Desktop/Javier/PHD_RIT/Latex/Proposal/Images/PIFstatL8.png}
%     % \vspace{1.5cm} 
%     \centerline{(b)}\medskip
%   \end{minipage}
%   \caption{Bright pixel determination using the PIF mask in ENVI. Statistics with the PIF mask applied for (a) Landsat 5 reflectance product (in reflectance units) and (b) statistics for Landsat 8 image (in radiances units). \label{fig:PIFstats} } 
% \end{figure}
% \vspace{1cm}
\begin{table}[!ht]
\caption{ PIF mean values for the Landsat 8 and Landsat 5 surface reflectance product. \label{tab:PIFvalues} } 
\centering
\begin{tabular}{c|c|c} 
 \bfseries{Band} & \bfseries{Landsat 8}& \bfseries{Landsat 5 reflectance}\\ 
 				 &	    $[W/m^2/sr/\mu m]$		   & \bfseries{product} \\ \hline \hline
 Coastal & 78.71 & N/A\\
 Blue    & 74.11 & 0.1041\\
 Green   & 61.11 & 0.1217\\
 Red     & 52.31 & 0.1310\\
 NIR     & 36.66 & 0.1565\\
 SWIR 1  & 9.50  & 0.1673\\ 
 SWIR 2  & 2.56  & 0.1558\\  
 \end{tabular}
\end{table}

\subsection{Solar Zenith Correction}
\autoref{fig:ZenithCorr} shows the mean values after applying the master PIF mask to the nine images of the Landsat 5 reflectance product mentioned above. Note that the PIF reflectance values for each scene are not the same. However, a high correlation between the mean of the reflectance values of the PIFs and the solar zenith angle of every image was found for each band. A linear relationship was determined for each band by applying a linear regression in MATLAB and the $R^2$ and root mean square error (RMSE) values were calculated as a way to measure this correlation. This linear relationship for each band has the form 
\begin{equation}
	y = m*x + y_0
	\label{eq:linear}
\end{equation}
where $x$ represents the solar zenith angle and $y$ the reflectance value. \autoref{fig:Band1Corr} shows the reflectance values versus the solar zenith angle for band 1 for the nine Landsat reflectance scenes and the calculated linear relationship. The values $m$ and $y_0$ found for each band are shown in \autoref{tab:ZenithCorr} along with the $R^2$ and RMSE values. It can been seen in \autoref{tab:ZenithCorr} that the $R^2$ values are larger than $90\%$ for all bands, which suggests there is a high correlation between the reflectance values and the solar zenith angle. As a conclusion, these results show that the material reflectance values remain constant over time and depend on the solar zenith angle of the sensor (i.e. the aggregate reflectance changes due to shadowing). 
\begin{figure}[!ht]
  	\centering
  	\includegraphics[height=7cm]{/Users/javier/Desktop/Javier/PHD_RIT/Latex/Proposal/Images/ZenithCorrection.eps}
  \caption{Mean values for nine scenes of the Landsat 5 reflectance product after applying the master PIF mask. \label{fig:ZenithCorr} } 
\end{figure}

\begin{figure}[htb]
  	\centering
  	\includegraphics[height=7cm]{/Users/javier/Desktop/Javier/PHD_RIT/Latex/Proposal/Images/ZenithCorrelation.eps}
  \caption{Linear regression between reflectance values and solar zenith angle for band 1 of the Landsat 5 reflectance product. \label{fig:Band1Corr} } 
\end{figure}

% \vspace{.5cm}
\begin{table}[!ht]
\caption{ Zenith angle correction parameters for the Landsat 5 reflectance product. \label{tab:ZenithCorr} } 
\centering
\begin{tabular}{l|c|c|c|c|c} 
 \bfseries{Band n} & \bfseries{$m$}      & \bfseries{$y_0$}    & \bfseries{$R^2$}     & \bfseries{$RMSE$} & $y(x=45^\circ)$   \\ \hline \hline
 Band 1 (Blue) 		& -0.000412 & 0.122631 & 0.937155 & 0.001705 &      0.1039\\
 Band 2 (Green) 	& -0.000634 & 0.147424 & 0.934344 & 0.002685 &      0.1186\\
 Band 3 (Red) 		& -0.000756 & 0.161421 & 0.976599 & 0.001869 &      0.1270\\
 Band 4 (NIR) 		& -0.001316 & 0.220031 & 0.906946 & 0.006733 &      0.1601\\
 Band 5 (SWIR 1) 	& -0.001148 & 0.217231 & 0.903702 & 0.005984 &      0.1650\\
 Band 6 (SWIR 2) 	& -0.001159 & 0.206725 & 0.929626 & 0.005096 &      0.1539\\  
 \end{tabular}
\end{table}

The Landsat 8 image has associated a particular solar zenith angle. The previous linear relationships allow us to simply estimate the reflectance values of the bright pixel for that particular solar zenith angle. This process is called solar zenith correction in this work. For example, the solar zenith angle for the 09-19-13 Landsat 8 scene is equal to $45^\circ$, therefore $x=45^\circ$ in \autoref{eq:linear}. The reflectance values for $x=45^\circ$ are shown in the last column of \autoref{tab:ZenithCorr} and plotted in \autoref{fig:ZenithCorr} in red asterisks.

Because the Landsat reflectance products was not available for Landsat 8 at the moment of writing this document, it was necessary to estimate a theoretical reflectance value for the coastal band for Landsat 5 in order to match with the Landsat 8 bands. To accomplish this, it was assumed that the coastal band would exhibit a similar trend to the blue and green bands. Hence, a straight-line that passes over the blue and green band values was used to extrapolate the value of the coastal band, as seen in \autoref{fig:Extrapol}, where the estimation of this reflectance value for the coastal band is shown at $443 [nm]$ and the straight-line is shown as a black solid line. It is expected the Landsat reflectance product will be available for Landsat 8 in the near future, so the previous step would not be necessary. Finally, the reflectance spectrum for the bright pixel including this coastal band estimation and the zenith angle correction is shown in \autoref{tab:brightref}. As was mentioned previously, the corresponding radiance spectrum for the bright pixel is obtained by applying the master PIF mask to the Landsat 8 image.
\vspace{.5cm}
\begin{table}[!ht]
\caption{ Reflectance spectrum for the bright pixel. \label{tab:brightref} } 
\centering
\begin{tabular}{l|c} 
 \bfseries{Band} & \bfseries{Reflectance values}\\ \hline \hline
 Band 1 (Coastal) 	&  0.0965 \\
 Band 2 (Blue) 		&  0.1039 \\
 Band 3 (Green) 	&  0.1186 \\
 Band 4 (Red) 		&  0.1270 \\
 Band 5 (NIR) 		&  0.1601 \\
 Band 6 (SWIR 1) 	&  0.1650 \\ 
 Band 7 (SWIR 2) 	&  0.1539 \\ 
 \end{tabular}
\end{table}

\begin{figure}[htb]
  	\centering
  	\includegraphics[height=7cm]{/Users/javier/Desktop/Javier/PHD_RIT/Latex/Proposal/Images/Extrapolation.eps}
  \caption{Extrapolation for the coastal band. Vertical lines represent the Landsat 8 bands (band center wavelength). \label{fig:Extrapol} } 
\end{figure}

\subsection{Black Pixel Determination}

The reflectance spectrum for the dark pixel is obtained using the extensively validated Hydrolight model\cite{MobleyHE}. Hydrolight is a radiative transfer numerical model written in Fortran that computes radiance distributions and derived quantities (e.g. irradiances, reflectances, K functions, etc.) for natural water bodies. Inherent Optical Properties (IOPs) and concentrations of CPAs were measured from a water sample taken in the field from a region of interest (ROI) in Lake Ontario, labeled as OntNS in \autoref{fig:0910913Sites}. The IOPs and concentrations from this particular location in the scene were input to the Case 2 model in Hydrolight and a spectral remote sensing reflectance ($R_{rs}(\lambda)$) was generated. The reflectance spectrum of the dark pixel is equal to $R_{rs}(\lambda)*\pi$. This Case 2 model is a generic four-component (these are pure water and CPAs) IOP model \cite{MobleyHEtech}. The Case 2 model in Hydrolight requires a specification in the concentration and IOPs of each component, one at a time. These IOPs include mass-specific absorption and scattering coefficient spectra, and scattering phase function for each component. The reflectance spectrum generated representing the OntNS site is shown in \autoref{fig:ELMpxsENVI}.(b) as a black solid line designated as HydroLight. The mean value of the pixels over the same ROI in the Landsat 8 image was used as the radiance values for the dark pixel.

\begin{figure}[htb]
  \centering
  \includegraphics[width=12cm]{/Users/javier/Desktop/Javier/PHD_RIT/Latex/Proposal/Images/groundtruth-sitenames-no-ends.jpg}
  \caption{Sites in the Rochester Embayment for the water sample collection on September, $19^{th}$, 2013.\label{fig:0910913Sites} } 
\end{figure}

As a result, \autoref{fig:ELMpxsENVI} shows the different spectra used to perform the model-based ELM, where four different spectra can be seen: one reflectance and one radiance spectrum for the bright pixel (obtained using the PIF extraction over the Landsat 5 reflectance product and Landsat 8 image, respectively), and one reflectance spectrum for the dark pixel (obtained from Hydrolight) and one radiance spectrum for the dark pixel (obtained from the statistics of a ROI over water in the Landsat 8 image). These spectra are used to atmospherically correct the Landsat 8 image using the ENVI Classic software\cite{ENVIUserGuide}. This is performed by using the ``Empirical Line'' algorithm from the ``Calibration Utilities'' in ENVI classic, where the Landsat 8 image is used as the input image, the reflectance spectra are used as the ``field spectra'', and the radiance spectra are used as the ``data spectra.'' The product of this process is an image in reflectance values. 

\begin{figure}[htb]
  \centering
  \includegraphics[width=14cm]{/Users/javier/Desktop/Javier/PHD_RIT/Latex/Proposal/Images/ELMpixelsENVI.pdf}
  \caption{Bright and Dark pixels used in ENVI to apply ELM. (a) Data spectra and (b) Field spectra \label{fig:ELMpxsENVI} } 
\end{figure}
%%%%%%%%%%%%%%%%%%%%%%%%%%%%%%%%%%%%%%%%%%%%%%%%%%
%%%%%%%%%%%%%%%%%%%%%%%%%%%%%%%%%%%%%%%%%%%%%%%%%%
\section{Results}
Preliminary results of the model-based ELM atmospheric correction methods for different water bodies in the Rochester Embayment area are shown in \autoref{fig:waterpxs}. This figure shows the spectrum of the water pixels in reflectance values after applying the model-based ELM atmospheric correction. These curves exhibit shapes that correspond with the shapes of typical water pixels. \autoref{fig:refcomp} shows water reflectance spectra as preliminary results from the model-based ELM method (solid lines) compared with results from a traditional ELM method (dashed lines) for four different ROIs in the Rochester Embayment area (Cranberry Pond, Long Pond, and nearshore and offshore of the Lake Ontario, labeled as Cranb, LongS, OntNS and OntOS in \autoref{fig:0910913Sites}, respectively). 

The traditional ELM method was performed with reflectance measurements taken in the field. A reflectance measurement taken of the sand of Charlotte Beach, Rochester, NY (labeled as SandDry in \autoref{fig:0910913Sites}) was used for the bright pixel while a reflectance measurement taken at the site OntNS was used for the dark pixel. The radiance values were taken from the corresponding ROIs in the Landsat 8 image. 

As can be seen in \autoref{fig:refcomp}, the atmospheric correction algorithm proposed in this study exhibits less than one percent reflectance unit ($<0.01\Rightarrow <1\%$) of difference in comparison to the results from the traditional ELM algorithm.

% \vspace{-.3cm}
\begin{figure}[htb]
  \begin{minipage}[c]{0.48\linewidth}
    \centering
      \includegraphics[height=6cm]{/Users/javier/Desktop/Javier/PHD_RIT/ConferencesAndApplications/NESSF14/latex/WaterPixels_2.eps}
      \caption{Water pixel spectra after applying the model-based ELM atmospheric correction method.}
      \label{fig:waterpxs}
    % \vspace{1.5cm}
    % \centerline{(a)}\medskip
  \end{minipage}
  \hfill
  \begin{minipage}[d]{0.48\linewidth}
    \centering
      \includegraphics[height=6cm]{/Users/javier/Desktop/Javier/PHD_RIT/ConferencesAndApplications/NESSF14/latex/WaterPixelComparisonELMELMbased}
      \caption{Comparison between traditional ELM (dashed lines) and model-based ELM (solid lines).}
      \label{fig:refcomp}
    % \vspace{1.5cm}
    % \centerline{(b)}\medskip
  \end{minipage}
  % \caption{Water pixel spectra: (a) after atmospheric correction and (b) comparison with traditional ELM method.}
\end{figure}

%%%%%%%%%%%%%%%%%%%%%%%%%%%%%%%%%%%%%%%%%%%%%%%%%%
%%%%%%%%%%%%%%%%%%%%%%%%%%%%%%%%%%%%%%%%%%%%%%%%%%
\section{Conclusion}
In this paper we have described an atmospheric correction algorithm proposed for Landsat 8. This algorithm tries to avoid the need for field data acquisition at every satellite overpass by using PIF extraction for the reflectance values for the bright pixel. As for the dark pixel, the Hydrolight model is used to generate a reflectance spectrum where IOPs and concentration are available or can be reasonable estimated. It was shown that this algorithm produces results comparable with a traditional ELM method. The differences are less than one percent reflectance unit ($<0.01\Rightarrow <1\%$) between the traditional ELM method and the model-based ELM method. While small, these differences could still have a significant impact on application such as water constituents retrieval so there is still need for improvement. A further validation with ground-truth data is anticipated for the future.

\section*{ACKNOWLEDGMENTS} 
The author would like to extend a special thanks to the United States Geological Survey (USGS) for its sponsorship that has made this effort possible.
% %%%%%%%%%%%%%%%%%%%%%%%%%%%%%%%%%%%%%%%%%%%%%%%%%%%%%%%%%%%%%
% \section{INTRODUCTION}
% \label{sec:intro}  % \label{} allows reference to this section

% This document shows the desired format and appearance of a manuscript prepared for the Proceedings of the SPIE.\footnote{The basic format was developed in 1995 by Rick Herman (SPIE) and Ken Hanson (Los Alamos National Lab.).} It is prepared using LaTeX2e\cite{Lamport94} with the class file {\tt spie.cls}.  The LaTeX source file used to create this document is {\tt article.tex}, which contains important formatting information embedded in it.  These files are available on the Internet at {\tt http://home.lanl.gov/kmh/spie/}.  The font used throughout is the LaTeX default font, Computer Modern Roman, which is equivalent to the Times Roman font available on many systems.  If this font is not available, use a similar serif font.  Normal text has a font size of 10 points\footnote{Font sizes are specified in points, abbreviated pt., which is a unit of length.  One inch = 72.27 pt.; one cm = 28.4 pt.} for which the actual height of a capital E is about 2.4 mm (7 pt.) and the line-to-line spacing is about 4.2 mm (12 pt.).  The font attributes for other parts of the manuscript, summarized in Table~\ref{tab:fonts}, are described in the following sections.  Normal text should be justified to both the left and right margins.  Appendix~\ref{sec:latex} has information about PostScript fonts.

% To be properly reproduced in the Proceedings, all text and figures must fit inside a rectangle 6.75-in.\ wide by 8.75-in.\ high or 17.15 cm by 22.23 cm.  The text width and height are set in {\tt spie.cls} to match this requirement.
% %% This table is carefully placed in the source file to make 
% %% it appear at bottom of page, but above the footnotes.  
% %% Use of [h] in following command forces table to appear "here".
% \begin{table}[h]
% \caption{Fonts sizes to be used for various parts of the manuscript.  All fonts are Computer Modern Roman or an equivalent.  Table captions should be centered above the table.  When the caption is too long to fit on one line, it should be justified to the right and left margins of the body of the text.} 
% \label{tab:fonts}
% \begin{center}       
% \begin{tabular}{|l|l|} %% this creates two columns
% %% |l|l| to left justify each column entry
% %% |c|c| to center each column entry
% %% use of \rule[]{}{} below opens up each row
% \hline
% \rule[-1ex]{0pt}{3.5ex}  Article title & 16 pt., bold, centered  \\
% \hline
% \rule[-1ex]{0pt}{3.5ex}  Author names and affiliations & 12 pt., normal, centered   \\
% \hline
% \rule[-1ex]{0pt}{3.5ex}  Section heading & 11 pt., bold, centered (all caps)  \\
% \hline
% \rule[-1ex]{0pt}{3.5ex}  Subsection heading & 11 pt., bold, left justified  \\
% \hline
% \rule[-1ex]{0pt}{3.5ex}  Sub-subsection heading & 10 pt., bold, left justified  \\
% \hline
% \rule[-1ex]{0pt}{3.5ex}  Normal text & 10 pt., normal  \\
% \hline
% \rule[-1ex]{0pt}{3.5ex}  Figure and table captions & \, 9 pt., normal \\
% \hline
% \rule[-1ex]{0pt}{3.5ex}  Footnote & \, 9 pt., normal \\
% \hline 
% \end{tabular}
% \end{center}
% \end{table} 
% The text should begin 1.00 in.\ or 2.54 cm from the top of the page.  The right and left margins should be 0.875~in.\ or 2.22 cm for US letter-size paper (8.5 in.\ by 11 in.) or 1.925 cm for A4 paper (210 mm by 297 mm) to horizontally center the text on the page.  See Appendix~\ref{sec:latex} for guidance regarding paper-size specification. 

% Authors are encouraged to follow the principles of sound technical writing, as described in Refs.~\citenum{Alred03} and \citenum{Perelman97}, for example.  Many aspects of technical writing are addressed in the {\em AIP Style Manual}, published by the American Institute of Physics.  It is available on line at {\tt http://www.aip.org/pubservs/style/4thed/toc.html} or {\tt http://public.lanl.gov/kmh/AIP\verb+_+Style\verb+_+4thed.pdf}. A spelling checker is helpful for finding misspelled words. 

% An author may use this LaTeX source file as a template by substituting his/her own text in each field.  This document is not meant to be a complete guide on how to use LaTeX.  For that, refer to books on LaTeX usage, such as the definitive work by Lamport\cite{Lamport94} or the very useful compendium by Mittelbach et al.\cite{Mittelbach04}

% %%%%%%%%%%%%%%%%%%%%%%%%%%%%%%%%%%%%%%%%%%%%%%%%%%%%%%%%%%%%%
% \section{PARTS OF MANUSCRIPT} 

% This section describes the normal structure of a manuscript and how each part should be handled.  The appropriate vertical spacing between various parts of this document is achieved in LaTeX through the proper use of defined constructs, such as \verb|\section{}|.  In LaTeX, paragraphs are separated by blank lines in the source file. 

% At times it may be desired, for formatting reasons, to break a line without starting a new paragraph.  This situation may occur, for example, when formatting the article title, author information, or section headings.  Line breaks are inserted in LaTeX by entering \verb|\\| or \verb|\linebreak| in the LaTeX source file at the desired location.  

% %%%%%Sometimes it is necessary to precede the double slash 
% %%%%%by \verb|\protect| to get the desired result, 
% %%%%%for example, in article titles.

% %%-----------------------------------------------------------
% \subsection{Title and Author Information} 
% \label{sec:title}

% The article title appears centered at the top of the first page.  The title font is 16 point, bold.  The rules for capitalizing the title are the same as for sentences; only the first word, proper nouns, and acronyms should be capitalized.  Avoid using acronyms in the title.  Keep in mind that people outside your area of expertise might read your article.  Appendix~\ref{sec:misc} contains more about acronyms.

% The list of authors immediately follows the title after a blank vertical space of about 7 mm.  The font is 12 point, normal with each line centered.  The authors' affiliations and addresses follow the author list after another blank space of about 4 mm, also in 12-point, normal font and centered.  Do not use acronyms in affiliations and addresses. For multiple affiliations, each affiliation should appear on a new line, separated from the following address by a semicolon.  Italicized superscripts may be used to identify the correspondence between the authors and their respective affiliations.  Further author information, such as e-mail address, complete postal address, and web-site location, may be provided in a footnote by using \verb|\authorinfo{}|, as demonstrated above.

% When the abbreviated title or author information is too long to fit on one line, multiple lines may be used; insert line breaks appropriately to achieve a visually pleasing format.  The proper spacing of one and one-half lines between the title, author list, and their affiliations is achieved with the command \verb|\skiplinehalf| defined in {\tt spie.cls}.

% %%-----------------------------------------------------------
% \subsection{Abstract and Keywords} 
% The title and author information is immediately followed by the Abstract. The Abstract should concisely summarize the key findings of the paper.  It should consist of a single paragraph containing no more than 200 words.  The Abstract does not have a section number.  A list of up to ten keywords should immediately follow the Abstract after a blank line.  These keywords will be included in a searchable database at SPIE.

% %%-----------------------------------------------------------
% \subsection{Body of Paper} 
% The body of the paper consists of numbered sections that present the main findings.  These sections should be organized to best present the material.  See Sec.~\ref{sec:sections} for formatting instructions.

% %%-----------------------------------------------------------
% \subsection{Appendices} 
% Auxiliary material that is best left out of the main body of the paper, for example, derivations of equations, proofs of theorems, and details of algorithms, may be included in appendices.  Appendices are enumerated with uppercase Latin letters in alphabetic order, and appear just before the Acknowledgments and References.

% %%-----------------------------------------------------------
% \subsection{Acknowledgments} 
% In the Acknowledgments section, appearing just before the References, the authors may credit others for their guidance or help.  Also, funding sources may be stated.  The Acknowledgments section does not have a section number.

% %%-----------------------------------------------------------
% \subsection{References} 
% The References section lists books, articles, and reports that are cited in the paper.  It does not have a section number.  The references are numbered in the order in which they are cited.  Examples of the format to be followed are given at the end of this document.  

% The reference list at the end of this document is created using BibTeX, which looks through the file {\tt report.bib} for the entries cited in the LaTeX source file.  The format of the reference list is determined by the bibliography style file {\tt spiebib.bst}, as specified in the \verb|\bibliographystyle{spiebib}| command.  Alternatively, the references may be directly formatted in the LaTeX source file.

% For books\cite{Lamport94,Alred03,Goossens97} the listing includes the list of authors, book title (in italics), page or chapter numbers, publisher, city, and year of publication.  A reference to a journal article\cite{Metropolis53} includes the author list, title of the article (in quotes), journal name (in italics, properly abbreviated), volume number (in bold), inclusive page numbers, and year.  By convention\cite{Lamport94}, article titles are capitalized as described in Sec.~\ref{sec:title}.  A reference to a proceedings paper or a chapter in an edited book\cite{Gull89a} includes the author list, title of the article (in quotes), conference name (in italics), if appropriate, editors, volume or series title (in italics), volume number (in bold), if applicable, inclusive page numbers, publisher, city, and year.  References to an article in the SPIE Proceedings may include the conference name, as shown in Ref.~\citenum{Hanson93c}.

% Citations to the references are made using superscript numerals, as demonstrated in the preceding paragraph.  One may also directly refer to a reference within the text, e.g., ``as shown in Ref.~\citenum{Metropolis53} ..." 

% %%-----------------------------------------------------------
% \subsection{Footnotes} 
% Footnotes\footnote{Footnotes are indicated as superscript symbols to avoid confusion with citations.} may be used to provide auxiliary information that doesn't need to appear in the text, e.g., to explain measurement units.  They should be used sparingly, however.  

% Only nine footnote symbols are available in LaTeX. If you have more than nine footnotes, you will need to restart the sequence using the command  \verb|\footnote[1]{Your footnote text goes here.}|. If you don't, LaTeX will provide the error message {\tt Counter too large.}, followed by the offending footnote command.

% %%%%%%%%%%%%%%%%%%%%%%%%%%%%%%%%%%%%%%%%%%%%%%%%%%%%%%%%%%%%%
% \section{SECTION FORMATTING} \label{sec:sections}

% Section headings are centered and formatted completely in uppercase 11-point bold font.  Sections should be numbered sequentially, starting with the first section after the Abstract.  The heading starts with the section number, followed by a period.  In LaTeX, a new section is created with the \verb|\section{}| command, which automatically numbers the sections.

% Paragraphs that immediately follow a section heading are leading paragraphs and should not be indented, according to standard publishing style\cite{Lamport94}.  The same goes for leading paragraphs of subsections and sub-subsections.  Subsequent paragraphs are standard paragraphs, with 14-pt.\ (5 mm) indentation.  An extra half-line space should be inserted between paragraphs.  In LaTeX, this spacing is specified by the parameter \verb|\parskip|, which is set in {\tt spie.cls}.  Indentation of the first line of a paragraph may be avoided by starting it with \verb|\noindent|.
 
% %%-----------------------------------------------------------
% \subsection{Subsection Attributes} 

% The subsection heading is left justified and set in 11-point, bold font.  Capitalization rules are the same as those for book titles.  The first word of a subsection heading is capitalized.  The remaining words are also capitalized, except for minor words with fewer than four letters, such as articles (a, an, and the), short prepositions (of, at, by, for, in, etc.), and short conjunctions (and, or, as, but, etc.).  Subsection numbers consist of the section number, followed by a period, and the subsection number within that section.  

% %%-----------
% \subsubsection{Sub-subsection attributes} 
% The sub-subsection heading is left justified and its font is 10 point, bold.  Capitalize as for sentences.  The first word of a sub-subsection heading is capitalized.  The rest of the heading is not capitalized, except for acronyms and proper names.  

% %%%%%%%%%%%%%%%%%%%%%%%%%%%%%%%%%%%%%%%%%%%%%%%%%%%%%%%%%%%%%
% \section{FIGURES AND TABLES} 

% Figures are numbered in the order of their first citation.  They should appear in numerical order and on or after the same page as their first reference in the text.  Alternatively, all figures may be placed at the end of the manuscript, that is, after the Reference section.  It is preferable to have figures appear at the top or bottom of the page.  Figures, along with their captions, should be separated from the main text by at least 0.2 in.\ or 5 mm.  

% Figure captions are centered below the figure or graph.  Figure captions start with the figure number in 9-point bold font, followed by a period; the text is in 9-point normal font; for example, ``{\footnotesize{Figure 3.}  Original image...}".  See Fig.~\ref{fig:example} for an example of a figure caption.  When the caption is too long to fit on one line, it should be justified to the right and left margins of the body of the text.  

% Tables are handled identically to figures, except that their captions appear above the table. 
% %%  Use following command to specify that graphics file is in 
% %%  a directory other than this LaTeX source file.
% %%  Note use of / to separate subdirectories, for UNIX and Windows OS.
% %%\graphicspath{{H:/HANSON/SPIESTY/}}
% %% tabular environment useful for creating an array of images  
% %-------------
%    \begin{figure}
%    \begin{center}
%    \begin{tabular}{c}
%    \includegraphics[height=7cm]{mcr3b.eps}
%    \end{tabular}
%    \end{center}
%    \caption[example] 
% %>>>> use \label inside caption to get Fig. number with \ref{}
%    { \label{fig:example} 
% Figure captions are used to describe the figure and help the reader understand it's significance.  The caption should be centered underneath the figure and set in 9-point font.  It is preferable for figures and tables to be placed at the top or bottom of the page. LaTeX tends to adhere to this standard.}
%    \end{figure} 
% %-------------

% For further details concerning the inclusion of grayscale and color images, consult SPIE's {\it Author Guide for Publication and Presentation}.
 
% %%%%%%%%%%%%%%%%%%%%%%%%%%%%%%%%%%%%%%%%%%%%%%%%%%%%
% \appendix    %>>>> this command starts appendixes
% %%%%%%%%%%%%%%%%%%%%%%%%%%%%%%%%%%%%%%%%%%%%%%%%%%%%
% \section{MISCELLANEOUS FORMATTING DETAILS} \label{sec:misc}

% It is often useful to refer back (or forward) to other sections in the article.  Such references are made by section number.  When a section reference starts a sentence, Section is spelled out; otherwise use its abbreviation, for example, ``In Sec.~2 we showed..." or ``Section~2.1 contained a description...".  References to figures, tables, and theorems are handled the same way.

% At the first occurrence of an acronym, spell it out, followed by the acronym in parentheses, e.g., noise power spectrum (NPS).  
 
% %%-----------------------------------------------
% \subsection{Formatting Equations} 
% Equations may appear in line with the text, if they are simple, short, and not of major importance; e.g., $\beta = b/r$.  Important equations appear on their own line.  Such equations are centered.  For example, ``The expression for the minus-log-posterior is
% 	\begin{equation}
% 	\label{eq:alpha}
% \phi = |{\rm\bf y} - {\rm\bf A}{\rm\bf x}|^2 + \alpha \log p({\rm\bf x}) \, ,
% 	\end{equation}
% where $\alpha$ determines the strength of ..."  Principal equations are numbered, with the equation number placed within parentheses and right justified.  

% Equations are considered to be part of a sentence and should be punctuated accordingly. In the above example, a comma follows the equation because the next line is a subordinate clause.  If the equation ends the sentence, a period should follow the equation.  The line following an equation should not be indented unless it is meant to start a new paragraph.  Indentation after an equation is avoided in LaTeX by not leaving a blank line between the equation and the subsequent text.

% References to equations include the equation number in parentheses, for example, ``Equation~(\ref{eq:alpha}) shows ..." or ``Combining Eqs.~(2) and (3), we obtain..."  Using a tilde in the LaTeX source file between two characters avoids unwanted line breaks.

% %%-----------------------------------------------------------
% \subsection{Formatting Theorems} 

% To include theorems in a formal way, the theorem identification should appear in a 10-point, bold font, left justified and followed by a period.  The text of the theorem continues on the same line in normal, 10-point font.  For example, 

% \noindent{\bf Theorem 1.} For any unbiased estimator...

% Formal statements of lemmas and algorithms receive a similar treatment.

% %%%%%%%%%%%%%%%%%%%%%%%%%%%%%%%%%%%%%%%%%%%%%%%%%%%%
% \section{SOME LATEX GUIDANCE} \label{sec:latex}

% %%-----------------------------------------------------------
% \subsection{Margins and PostScript Fonts}
 
% Manuscripts submitted electronically to as PostScript (PS) files must have the correct margins. LaTeX margins are related to the document's paper size. The paper size is set at two separate places in the process of creating a PS file. The first place is in {\tt latex}. The default in {\tt article.tex}, on which {\tt spie.cls} is based, is USA letter paper. To format a document for A4 paper, the first line of the LaTeX source file should be \verb|\documentclass[a4paper]{spie}|.   

% The output of the LaTeX utility is a file with the extension DVI (for Device Independent), which encodes the formatted document.  The application DVIPS is typically used to convert the DVI file to a PS file.  DVIPS has its own default paper size, which can be overridden with the option ``{\tt -t letter}" or ``{\tt -t a4}".  
% If the foregoing steps do not produce the correct top margin, you can move the text lower on the page (by 9 mm) with the command \verb|\addtolength{\voffset}{9mm}|, placed right after the \verb|\documentclass| command, for example.

% Another DVIPS option specifies the incorporation of (scalable) PostScript Type 1 fonts in its output PS file. This feature is important for obtaining a subsequent PDF file that will be clearly displayed on a computer monitor by Adobe Acrobat Reader.  The option ``{\tt -P pdf}" makes DVIPS include these fonts in its output PS file.

% %%-----------------------------------------------------------
% \subsection{Bold Math Symbols} 

% The math package from the American Mathematical Society allows one to easily produce bold math symbols, well beyond what is available in LaTeX. It also provides many useful capabilities for creating elaborate mathematical expressions. You need to load the AMS math package near the top of the LaTeX source file, right after the \verb+\documentclass+ command:\\[1ex]
% \verb+\usepackage[]{amsmath}+ \\[1ex]
% Then for bold math symbols use \verb+\boldsymbol+ in equations, e.g., 
% \verb+$\boldsymbol{\pi}$+ 
% yields a bold pi.  You can make it easier to use by defining a command:\\[1ex]
% \verb+\newcommand{\bm}[1]{\boldsymbol{#1}}+ \\[1ex]
% and then using it like so \verb+$\bm{\pi}$+.

% Not all math symbols are available in bold.  In a pinch, you can use \verb+\pmb+ ("poor man's bold"), which is defined in \verb+amsmath+. This command approximates a bold character with a superposition of several, slightly displaced unbold characters.

% If you want a Greek symbol in the article title, it should be both larger and bold. The easiest thing is to load the AMS math package as described above. 
% Then, in the title, use something like:\\[1ex]
% \verb+\title{Estimation of {\LARGE$\boldsymbol\alpha$} by a Monte Carlo technique}+ \\[1ex]
% Note that the command to create the alpha character is enclosed within braces to form a self-contained environment. The use of \verb+\LARGE+ in this example may not be needed when using nondefault font packages, such as the {\tt times} package, because of how the article title is handled in {\tt spie.cls}.

% %%-----------------------------------------------------------
% \subsection{Uppercase letters and special symbols in BibTex} 

% BibTeX tries to enforce standard publishing rules regarding article titles and authors' names; it sometimes changes uppercase letters to lower case. BibTeX also has trouble with umlauts, generally created in LaTeX with \verb+\"{o}+, because it is looking for the \verb+"+ to end the input line. 

% The general rule for overriding LaTeX's and BibTex's reinterpretation of your input text is to put the items you wish to be unchanged in braces. Thus, to obtain an umlaut in an author's name or in an article title, or to force an uppercase letter, do something like the following: \\[1ex]
% \verb+ @article{Kaczmarz37,+ \\ 
% \verb+ author = "S. Kaczmarz",+ \\ 
% \verb+ title  = "Angen{\"{a}}hrte {A}ufl{\"{o}}sung von {S}ystemen linearer {G}leichungen",+ \\ 
% \verb+ journal= "Bull. Acad. Polon. Sci. Lett.",+ \\ 
% \verb+ volume = "A35",+ \\ 
% \verb+ pages  = "355-357",+ \\	
% \verb+ year   = "1937"	} + \\[1ex]
% This example shows the use of both umlauts and uppercase letters.
% %%%%%%%%%%%%%%%%%%%%%%%%%%%%%%%%%%%%%%%%%%%%%%%%%%%%%%%%%%%%%
% \acknowledgments     %>>>> equivalent to \section*{ACKNOWLEDGMENTS}       
 
% This unnumbered section is used to identify those who have aided the authors in understanding or accomplishing the work presented and to acknowledge sources of funding.  

% %%%%%%%%%%%%%%%%%%%%%%%%%%%%%%%%%%%%%%%%%%%%%%%%%%%%%%%%%%%%%
% %%%%% References %%%%%

% \bibliography{report}   %>>>> bibliography data in report.bib
\bibliography{/Users/javier/Desktop/Javier/PHD_RIT/Latex/javier_bib}
\bibliographystyle{spiebib}   %>>>> makes bibtex use spiebib.bst

\end{document} 
