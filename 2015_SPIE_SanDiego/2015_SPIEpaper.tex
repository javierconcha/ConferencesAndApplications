%  article.tex (Version 3.3, released 19 January 2008)
%  Article to demonstrate format for SPIE Proceedings
%  Special instructions are included in this file after the
%  symbol %>>>>
%  Numerous commands are commented out, but included to show how
%  to effect various options, e.g., to print page numbers, etc.
%  This LaTeX source file is composed for LaTeX2e.

%  The following commands have been added in the SPIE class 
%  file (spie.cls) and will not be understood in other classes:
%  \supit{}, \authorinfo{}, \skiplinehalf, \keywords{}
%  The bibliography style file is called spiebib.bst, 
%  which replaces the standard style unstr.bst.  

\documentclass[draft]{spie}  %>>> use for US letter paper
%%\documentclass[a4paper]{spie}  %>>> use this instead for A4 paper
%%\documentclass[nocompress]{spie}  %>>> to avoid compression of citations
%% \addtolength{\voffset}{9mm}   %>>> moves text field down
%% \renewcommand{\baselinestretch}{1.65}   %>>> 1.65 for double spacing, 1.25 for 1.5 spacing 
%  The following command loads a graphics package to include images 
%  in the document. It may be necessary to specify a DVI driver option,
%  e.g., [dvips], but that may be inappropriate for some LaTeX 
%  installations. 
\usepackage[]{graphicx}
\usepackage[percent]{overpic}
% \usepackage[authoryear]{natbib}

\title{Validation of the MoB-ELM Atmospheric Correction Algorithm for Landsat 8 over Case 2 Waters} 

%>>>> The author is responsible for formatting the 
%  author list and their institutions.  Use  \skiplinehalf 
%  to separate author list from addresses and between each address.
%  The correspondence between each author and his/her address
%  can be indicated with a superscript in italics, 
%  which is easily obtained with \supit{}.

\author{Javier A. Concha and John R. Schott
\skiplinehalf
Digital Imaging and Remote Sensig Lab\\Chester F. Carlson Center for Imaging Science\\Rochester Institute of Technology\\ 54 Lomb Memorial Dr., Rochester, NY 14623, USA\\
}

%>>>> Further information about the authors, other than their 
%  institution and addresses, should be included as a footnote, 
%  which is facilitated by the \authorinfo{} command.

\authorinfo{Further author information: (Send correspondence to Javier A. Concha)\\ E-mail: jxc4005@rit.edu, Telephone: 1-(585) 290-3145}%%>>>> when using amstex, you need to use @@ instead of @
 

%%%%%%%%%%%%%%%%%%%%%%%%%%%%%%%%%%%%%%%%%%%%%%%%%%%%%%%%%%%%% 
%>>>> uncomment following for page numbers
% \pagestyle{plain}    
%>>>> uncomment following to start page numbering at 301 
%\setcounter{page}{301} 

%% Added 07-29-15
\usepackage{multirow}% http://ctan.org/pkg/multirow
\usepackage{hhline}% http://ctan.org/pkg/hhline
\usepackage{epstopdf}
\epstopdfsetup{update} % only regenerate pdf files when eps file is newer
\usepackage{amsmath,epsfig}
\usepackage{hyperref}
 
  \begin{document} 
  \maketitle 

%%%%%%%%%%%%%%%%%%%%%%%%%%%%%%%%%%%%%%%%%%%%%%%%%%%%%%%%%%%%% 
\begin{abstract}
Landsat 8 is a promising candidate to address the remote sensing of inland and coastal waters (Case 2 waters) due to its improved signal-to-noise ratio (SNR), spectral resolution, bit quantization, and high spatial resolution. Atmospheric correction is essential for remote sensing of water since the signal from the water reaching the sensor is small compared to atmospheric scattering. Standard atmospheric correction algorithms fail over highly turbid Case 2 waters because the black pixel assumption, i.e. the signal leaving the water is zero beyond near infrared (NIR), is not always satisfied. We developed a new atmospheric correction algorithm, the model-based ELM (MoB-ELM), for Landsat 8 imagery based on the empirical line method (ELM) that does not rely on the black pixel assumption. This algorithm uses pseudo-invariant features (PIF) from the image, ground-truth data, and water-leaving reflectances from an in-water radiative transfer model (Hidrolight) to determine reflectance and radiance values of the bright and dark pixels used in the ELM method.  We compare the results with in situ remote sensing reflectance measurements for different water bodies that exhibit a range of optical properties. We calculate reflectance errors for each band taking the in situ data as ground-truth, and then compare them to results from standard atmospheric correction algorithms. These reflectance errors are small in all the visible bands for a wide range of concentrations. These results show that our atmospheric correction algorithm allows one to use Landsat 8 to study Case 2 waters as an alternative to traditional ocean color satellites (e.g. MODIS, SeaWiFS). 
\end{abstract}

%>>>> Include a list of keywords after the abstract 
\keywords{Atmospheric Correction, Case 2 Water, Landsat 8, Inland and Coastal Water, Empirical Line Method, Operational Land Imager, OLI, Ocean Color}


%%%%%%%%%%%%%%%%%%%%%%%%%%%%%%%%%%%%%%%%%%%%%%%%%%%%%%%%%%%%%
\section{INTRODUCTION}
\label{sec:intro}  % \label{} allows reference to this section

Concentration range for Chl in the dataset used for the parameters of the band ratio algorithm does not contain high values $> 50mg/L?$

%%%%%%%%%%%%%%%%%%%%%%%%%%%%%%%%%%%%%%%%%%%%%%%%%%%%%%%%%%%%%
\section{METHODS}
\label{sec:methods}
% -----------------------------------------------------------
\subsection{Satellite Data}
The Landsat 8 satellite was launched in February of 2013 as a joint initiative between the U.S. Geological Survey (USGS) and NASA. Two push-broom instruments are onboard the Landsat 8 satellite: the Operational Land Imager (OLI) and the Thermal Infrared Sensor (TIRS). The satellite has a 16-day repeat cycle with a scene size of 170 km north-south by 183 km east-west. The instrument of interest in this study is OLI. OLI is a optical sensor with a total of nine bands; four bands in the visible (band 1-4), one near infrared (NIR) band (band 5), two short-wave infrared (SWIR) bands (band 6-7), one panchromatic band (band 8) and one NIR band for the detection of cirrus clouds (band 9). Landsat 8 has a higher signal-to-noise ratio (SNR) and two new bands (band 1 and band 9) compared with its predecessors (e.g. Landsat 7’s ETM+ sensor). The specification of te OLI's bands are shown in \autoref{tab:L8specs}. The Landsat 8 image used in this study was acquired on 09-19-2013 (scene LC80160302013262LGN00). This Landsat 8 Level 1T data product was downloaded from the USGS EarthExplorer website (\url{https://earthexplorer.usgs.gov/}) in GeoTIFF format. The product was in digital numbers (DNs) and was converted to spectral radiance at the sensor's aperture using the radiometric calibration parameters in the metadata file (\_MTL.txt) provided with the product.


\begin{table}[!ht]
\caption{ OLI band's specifications. \label{tab:L8specs} } 
\vspace{0.2cm}
\centering
\begin{tabular}{lccccl} 
 % \bfseries{Band n} & \bfseries{$m$}      & \bfseries{$y_0$}    & \bfseries{$R^2$}     & \bfseries{$RMSE$} & $y(x=45^\circ)$   \\ \hline \hline
 \hline
Band  			&	Band 		& Band Center 	&	Band Width  &	GDS  	\\ 
      			&   Number 	    &	$[nm]$ 		&	$[nm]$		& $[m]$ 	\\ \hline \hline
Coastal/Aerosol & 	1 			&	443  		& 	16 			& 30	 	\\  	
Blue 			& 	2 			&	483  		& 	60 			& 30	 	\\  	
Green 			& 	3 			&	561  		& 	57 			& 30 	 	\\  	
Red 			& 	4 			&	655  		& 	37 			& 30	 	\\  	
NIR 			& 	5 			&	865  		& 	28			& 30	 	\\  	
SWIR 1 			& 	6 			&	1609 		& 	85 			& 30	 	\\  	
SWIR 2 			& 	7 			&	2201 		& 	187 		& 30	 	\\  	
Panchromatic 	&	8 			&	590  		& 	180 		& 15	 	\\  	
Cirrus 			&	9 			&	1375 		& 	20 			& 30	 	\\	\hline
 \end{tabular}	
\end{table}	
% -----------------------------------------------------------
\subsection{Study Area and Field Measurements}

The study area is the Rochester Embayment, Rochester, NY (latitude: $43^\circ18'$N and longitude: $76^\circ42'$W). \autoref{fig:RrsROIs130919} shows a portion of a Landsat 8 image including the study area. This study area was chosen because it exhibits a wide range of CPA concentrations, including some eutrophic water bodies (ponds) with high concentration of CPAs, and oligomesotrophic water bodies (Lake Ontario) with low concentration of CPAs. A field collection that includes water samples and remote-sensing reflectance ($R_{rs}$) measurements was conducted at the same time of the sensor overpass. The $R_{rs}$ measurements were performed using a SVC HR-1024i instrument\cite{SVCHR1024i} following the method described by Mobley\cite{Mobley:1999} for measuring the spectra of the downwelling irradiance $E_d$, the surface reflected sky radiance $L_s$, and the water-leaving radiance $L_w$ for each site. Then, the water samples were analyzed in the lab, following SeaWiFS protocols\cite{Mueller1995} for obtaining chlorophyll-{\it a} concentration ($C_a$) and total suspended solid concentration (TSS). \autoref{tab:Sites} shows the different site names, location and the $C_a$ and $TSS$ for each site for the collection on 09-19-2013. Note the difference in concentration levels between the ponds (i.e. LONGN, LONGS and CRANB) and the lake (i.e. ONTNS, ONTOS and ONTEX) samples.

% Site  &	$C_a$  &  	Latitude  &	Longitude
% ONTNS & 	~~0.48 &	43.272159 &	-77.538274 	
% ONTOS & 	~~0.96 &	43.308923 &	-77.540085 	
% ONTEX & 	~~1.68 &	43.244892 &	-77.536671 	
% RVRPI & 	~~2.88 &	43.259925 &	-77.601587 	
% RVRPL & 	~~0.48 &	43.270990 &	-77.592282 	
% LONGN & 	123.85 &	43.290836 &	-77.690662 	
% LONGS & 	112.76 &	43.289182 &	-77.696458 	
% CRANB & 	~64.08 &	43.299938 &	-77.692915 	
% BRADI & 	~19.22 &	43.313675 &	-77.717531 	
% BRADO & 	~~1.44 & 	43.325780 &	-77.706432 

\begin{table}[!ht]
\caption{ Different sites for the collection on 09-19-2013. \label{tab:Sites} } 
\vspace{0.2cm}
\centering
\begin{tabular}{lccccl} 
 % \bfseries{Band n} & \bfseries{$m$}      & \bfseries{$y_0$}    & \bfseries{$R^2$}     & \bfseries{$RMSE$} & $y(x=45^\circ)$   \\ \hline \hline
 \hline
Site  &    	Latitude  &	Longitude  &	$C_a$  	   &	$TSS$  	& Description	\\ 
      &    	          &			   &	$[mg/m^3]$ & $[g/m^3]$ 	& 	\\ \hline \hline
ONTNS &    	43.272159 &	-77.538274 & 	~~0.48 & ~1.60	 		& Lake Ontario near-shore	\\  	
ONTOS &    	43.308923 &	-77.540085 & 	~~0.96 & ~1.00	 		& Lake Ontario off-shore	\\  	
ONTEX &    	43.244892 &	-77.536671 & 	~~1.68 & ~0.70 	 		& Lake Ontario extra	\\  	
RVRPI &    	43.259925 &	-77.601587 & 	~~2.88 & ~2.10	 		& Genese River pier	\\  	
RVRPL &    	43.270990 &	-77.592282 & 	~~0.48 & ~1.00	 		& Genese River plume	\\  	
LONGN &    	43.290836 &	-77.690662 & 	123.85 & 48.00	 		& Long Pong north	\\  	
LONGS &    	43.289182 &	-77.696458 & 	112.76 & 46.00	 		& Long Pond south	\\  	
CRANB &    	43.299938 &	-77.692915 & 	~64.08 & 26.70	 		& Cranberry Pond	\\  	
BRADIN&    	43.313675 &	-77.717531 & 	~19.22 & 13.10	 		& inside Braddock bay	\\  	
BRADONT&	43.325780 &	-77.706432 & 	~~1.44 & ~2.00	  		& Braddock Bay, Lake Ontario side	\\  \hline
 \end{tabular}	
\end{table}	

\begin{figure}[htbp!]
  \centering
  \includegraphics[height=8.0cm]{./Images/ROI_RocEmbayment130919-eps-converted-to.pdf}
  \caption{Landsat 8 image acquired on 09-19-2015 (scene LC80160302013262LGN00) showing the study area, the Rochester Embayment. The labels indicate the sites of the field collection (Labels: ONTNS: Lake Ontario near-shore, ONTOS: Lake Ontario off-shore, ONTEX: Lake Ontario extra, RVRPIER: Genese River pier, RVRPLM: Genese River plume, LONGN: Long Pong north, LONGS: Long Pond south, CRANB: Cranberry Pond, BRADIN: inside Braddock Bay, and BRADONT: Braddock Bay, Lake Ontario side).\label{fig:RrsROIs130919} } 
\end{figure}

%%%%%%%%%%%%%%%%%%%%%%%%%%%%%%%%%%%%%%%%%%%%%%%%%%%%%%%%%%%%%
\section{ATMOSPHERIC CORRECTION}
\label{sec:atmcorr}  % \label{} allows reference to this section
% -----------------------------------------------------------
\subsection{MOB-ELM}
% -----------------------------------------------------------
\subsection{SEADAS-SWIR}
The bands used for the SeaDAS processing were the SWIR1 and SWIR2 because there is some contribution from the NIR band due the highly turbid water in the ponds and water-leaving radiance signal cannot be considered negligible.
% -----------------------------------------------------------
\subsection{ACOLITE-SWIR}
% -----------------------------------------------------------
\subsection{SeaDAS-MUMM}


%%%%%%%%%%%%%%%%%%%%%%%%%%%%%%%%%%%%%%%%%%%%%%%%%%%%%%%%%%%%%
\section{CPA RETRIEVAL}
% -----------------------------------------------------------
\subsection{Bio-Optical Algorithm}
% -----------------------------------------------------------
\subsection{LUT Approach}
%%%%%%%%%%%%%%%%%%%%%%%%%%%%%%%%%%%%%%%%%%%%%%%%%%%%%%%%%%%%%
\section{RESULTS AND DISCUSSION}
\label{sec:results}  % \label{} allows reference to this section

There were four different algorithms analyzed in this study. Three of them are based in the standard algorithms based on the methods developed by Gordon and Wang \cite{Gordon:1994}. The fourth algorithm is based on the algorithm developed by Concha and Schott\cite{Concha2014SPIE} and Gerace et al.\cite{Gerace:2012}.
% -----------------------------------------------------------
\subsection{Comparison of Rrs}

\begin{equation}
\label{eq:NRMSE}
	NRMSE =\frac{\sqrt{\frac{1}{N}\sum_{n=1}^N{\left[C_{ret}(n) - C_{mea}(n)\right]^2}}}{max\{C_{mea}(n)\} - min\{C_{mea}(n)\}}\times100 ~[\%]
\end{equation}

% Method 1    Method 2    wl    NegTool1  NegTool2    usable  rsq_SS      rsq_corr    slope   offset      R^2         N           RMSE
% Acolite     MoB-ELM     443   0.53      0.09        97      .0.8791     0.8828      0.9444  -0.0047     0.8791      144047     0.0054
% Acolite     SeaDAS      443   0.53      76.74       98      .0.7804     0.7924      1.1437  -0.0053     0.7804      145186     0.0038
% Acolite     MUMM        443   0.53      74.41       99      .0.8554     0.8606      1.0600  -0.0032     0.8554      147730     0.0026
% SeaDAS      MoB-ELM     443   43.98     0.05        55      .0.7637     0.7776      0.8664  -0.0007     0.7637      141563     0.0019
% SeaDAS      MUMM        443   43.98     42.64       56      .0.8456     0.8516      0.9175  +0.0018     0.8456      145315     0.0014
% MUMM        MoB-ELM     443   42.64     0.05        56      .0.7474     0.7634      0.8811  -0.0018     0.7474      144947     0.0029
% Acolite     MoB-ELM     483   0.51      0.00        97      .0.9302     0.9314      0.9353  -0.0030     0.9302      144077     0.0037
% Acolite     SeaDAS      483   0.51      76.36       98      .0.8739     0.8779      1.1269  -0.0041     0.8739      145692     0.0028
% Acolite     MUMM        483   0.51      74.29       99      .0.9175     0.9192      1.0693  -0.0024     0.9175      147837     0.0018
% SeaDAS      MoB-ELM     483   43.76     0.00        55      .0.8454     0.8514      0.8547  +0.0002     0.8454      142116     0.0014
% SeaDAS      MUMM        483   43.76     42.57       56      .0.9122     0.9141      0.9441  +0.0015     0.9122      145888     0.0012
% MUMM        MoB-ELM     483   42.57     0.00        56      .0.8408     0.8471      0.8639  -0.0007     0.8408      145138     0.0022
% Acolite     MoB-ELM     561   0.44      0.00        97      .0.9044     0.9067      1.0183  -0.0011     0.9044      144192     0.0011
% Acolite     SeaDAS      561   0.44      75.56       99      .0.8233     0.8311      1.0307  -0.0013     0.8233      146761     0.0013
% Acolite     MUMM        561   0.44      74.15       100     .0.8968     0.8995      0.9663  -0.0001     0.8968      148040     0.0007
% SeaDAS      MoB-ELM     561   43.30     0.00        55      .0.7828     0.7946      1.0043  +0.0001     0.7828      143288     0.0009
% SeaDAS      MUMM        561   43.30     42.49       57      .0.8728     0.8768      0.9374  +0.0011     0.8728      147116     0.0010
% MUMM        MoB-ELM     561   42.49     0.00        56      .0.8468     0.8527      1.0611  -0.0010     0.8468      145339     0.0009
% Acolite     MoB-ELM     655   0.49      0.00        97      .0.8592     0.8641      1.1703  -0.0017     0.8592      144108     0.0013
% Acolite     SeaDAS      655   0.49      76.22       98      .0.7090     0.7301      0.9628  -0.0011     0.7090      145798     0.0014
% Acolite     MUMM        655   0.49      74.15       99      .0.7880     0.7992      0.9700  -0.0008     0.7880      147956     0.0010
% SeaDAS      MoB-ELM     655   43.68     0.00        55      .0.7708     0.7840      1.2158  -0.0004     0.7708      142375     0.0007
% SeaDAS      MUMM        655   43.68     42.49       56      .0.7964     0.8067      1.0063  +0.0004     0.7964      146144     0.0007
% MUMM        MoB-ELM     655   42.49     0.00        56      .0.6841     0.7091      1.2403  -0.0008     0.6841      145340     0.0009   
 
\begin{table}[!ht]
\caption{ Comparison different methods for retrieving $R_{rs}$. \label{tab:RrsCompMethod} } 
\centering
\scriptsize
\begin{tabular}{cllcccccccc} 
 % \bfseries{Band n} & \bfseries{$m$}      & \bfseries{$y_0$}    & \bfseries{$R^2$}     & \bfseries{$RMSE$} & $y(x=45^\circ)$   \\ \hline \hline
Band		&   Method 1      &  Method 2	  &	Slope  	&	Offset  &	$R^2 $  &	N      	&	RMSE    &\multicolumn{2}{c}{$R_{rs}<0~[\%]$}   &   Used 	 \\ 
$[nm]$		&	  		      &  		 	  &	  		&			&	   		&	     	&$[mg/m^3]$ & Method 1   	& Method 2  		   & $[\%]$		 \\	\hline \hline
\multirow{6}{*}{443}&Acolite-SWIR&MoB-ELM     &	0.9444 	&	-0.0047 &	0.8791 	&	144047  &	0.0054  &  ~0.53     	& ~0.09      		   &   97   	 \\
			&   SeaDAS-SWIR   &  MoB-ELM      &	0.8664 	&	-0.0007 &	0.7637 	&	141563  &	0.0019  &  43.98    	& ~0.05      		   &   55   	 \\
			&   SeaDAS-MUMM   &  MoB-ELM      &	0.8811 	&	-0.0018 &	0.7474 	&	144947  &	0.0029  &  42.64    	& ~0.05      		   &   56   	 \\
			&   Acolite-SWIR  &  SeaDAS-SWIR  &	1.1437 	&	-0.0053 &	0.7804 	&	145186  &	0.0038  &  ~0.53     	& 76.74     		   &   98   	 \\
			&   Acolite-SWIR  &  SeaDAS-MUMM  &	1.0600 	&	-0.0032 &	0.8554 	&	147730  &	0.0026  &  ~0.53     	& 74.41     		   &   99   	 \\
			&   SeaDAS-SWIR   &  SeaDAS-MUMM  &	0.9175 	&	~0.0018 &	0.8456 	&	145315  &	0.0014  &  43.98    	& 42.64     		   &   56   	 \\  \hline
\multirow{6}{*}{448}&Acolite-SWIR&MoB-ELM     &	0.9353 	&	-0.0030 &	0.9302 	&	144077  &	0.0037  &  ~0.51     	& ~0.00      		   &   97   	 \\
			&   SeaDAS-SWIR   &  MoB-ELM      &	0.8547 	&	~0.0002 &	0.8454 	&	142116  &	0.0014  &  43.76    	& ~0.00      		   &   55   	 \\
			&   SeaDAS-MUMM   &  MoB-ELM      &	0.8639 	&	-0.0007 &	0.8408 	&	145138  &	0.0022  &  42.57    	& ~0.00      		   &   56   	 \\ 
			&   Acolite-SWIR  &  SeaDAS-SWIR  &	1.1269 	&	-0.0041 &	0.8739 	&	145692  &	0.0028  &  ~0.51     	& 76.36     		   &   98   	 \\
			&   Acolite-SWIR  &  SeaDAS-MUMM  &	1.0693 	&	-0.0024 &	0.9175 	&	147837  &	0.0018  &  ~0.51     	& 74.29     		   &   99   	 \\
			&   SeaDAS-SWIR   &  SeaDAS-MUMM  &	0.9441 	&	~0.0015 &	0.9122 	&	145888  &	0.0012  &  43.76    	& 42.57     		   &   56   	 \\ \hline
\multirow{6}{*}{561}&Acolite-SWIR&  MoB-ELM   &	1.0183 	&	-0.0011 &	0.9044 	&	144192  &	0.0011  &  ~0.44     	& ~0.00      		   &   97   	 \\
	 		&   SeaDAS-SWIR   &  MoB-ELM      &	1.0043 	&	~0.0001 &	0.7828 	&	143288  &	0.0009  &  43.30    	& ~0.00      		   &   55   	 \\
	 		&   SeaDAS-MUMM   &  MoB-ELM      &	1.0611 	&	-0.0010 &	0.8468 	&	145339  &	0.0009  &  42.49    	& ~0.00      		   &   56   	 \\
	 		&   Acolite-SWIR  &  SeaDAS-SWIR  &	1.0307 	&	-0.0013 &	0.8233 	&	146761  &	0.0013  &  ~0.44     	& 75.56     		   &   99   	 \\
	 		&   Acolite-SWIR  &  SeaDAS-MUMM  &	0.9663 	&	-0.0001 &	0.8968 	&	148040  &	0.0007  &  ~0.44     	& 74.15     		   &   100  	 \\
	  		&   SeaDAS-SWIR   &  SeaDAS-MUMM  &	0.9374 	&	~0.0011 &	0.8728 	&	147116  &	0.0010  &  43.30    	& 42.49     		   &   57   	 \\ \hline
\multirow{6}{*}{655}&Acolite-SWIR&MoB-ELM     &	1.1703 	&	-0.0017 &	0.8592 	&	144108  &	0.0013  &  ~0.49     	& ~0.00      		   &   97   	 \\
	 		&   SeaDAS-SWIR   &  MoB-ELM      &	1.2158 	&	-0.0004 &	0.7708 	&	142375  &	0.0007  &  43.68    	& ~0.00      		   &   55   	 \\
	 		&   SeaDAS-MUMM   &  MoB-ELM      &	1.2403 	&	-0.0008 &	0.6841 	&	145340  &	0.0009  &  42.49    	& ~0.00      		   &   56   	 \\ 
	 		&   Acolite-SWIR  &  SeaDAS-SWIR  &	0.9628 	&	-0.0011 &	0.7090 	&	145798  &	0.0014  &  ~0.49     	& 76.22     		   &   98   	 \\
	 		&   Acolite-SWIR  &  SeaDAS-MUMM  &	0.9700 	&	-0.0008 &	0.7880 	&	147956  &	0.0010  &  ~0.49     	& 74.15     		   &   99   	 \\
	 		&   SeaDAS-SWIR   &  SeaDAS-MUMM  &	1.0063 	&	~0.0004 &	0.7964 	&	146144  &	0.0007  &  43.68    	& 42.49     		   &   56   	 \\
 \end{tabular}
\end{table}

% \begin{figure}[htbp!]
%   \centering
%   \includegraphics[height=8.0cm]{./Images/Collated2013262_2_band_1_D.png}
%   \caption{$R_{rs}$ for the sites for the 09-19-2013 collection after applying the MoB-ELM atmospheric correction (Labels: LONGS: Long Pond south, CRANB: Cranberry Pond, ONTOS: Lake Ontario off-shore, and ONTNS: Lake Ontario near-shore).\label{fig:RrsROIs130919} } 
% \end{figure}

% \begin{overpic}[width=0.5\textwidth,grid,tics=10]{pictures/baum}
%  \put (20,85) {\huge$\displaystyle\gamma$}
% \end{overpic}
%^^^^^^^^^^^^^^^^^^^  FIGURE ^^^^^^^^^^^^^^^^^^^^^^^^^^^^^^^^^^^^^^^^^^^^
\begin{figure}[htbp!]
	\begin{minipage}[c]{0.48\linewidth}
  		\centering
  		\begin{overpic}[trim=0 200 0 0,clip,width=7.5cm]{./Images/subset_0_of_Collocated13262_ACOSWIR_MOB_SEA_Rrs_443MOBdivpi}
  		\put (5,5) {MOB-ELM}
  		\end{overpic}
  	\end{minipage}
  	\hfill
	\begin{minipage}[c]{0.48\linewidth}
  		\centering
  		\begin{overpic}[trim=0 200 0 0,clip,width=7.5cm]{./Images/subset_0_of_Collocated13262_ACOSWIR_MOB_SEA_Rrs_443ACO}
  		\put (5,5) {Acolite-SWIR}
  		\end{overpic}
  	\end{minipage}

	\begin{minipage}[c]{0.48\linewidth}
  		\centering
  		\begin{overpic}[trim=0 200 0 0,clip,width=7.5cm]{./Images/subset_0_of_Collocated13262_ACOSWIR_MOB_SEA_Rrs_443SEA}
  		\put (5,5) {SeaDAS-SWIR}
  		\end{overpic}
  	\end{minipage}
  	\hfill
	\begin{minipage}[c]{0.48\linewidth}
  		\centering
  		\begin{overpic}[trim=30 170 40 150,clip,width=7.5cm]{./Images/Collocated13262_ACOSWIR_MOB_SEA5x5_MUMM45_Rrs_443_MUMM45}
  		\put (5,5) {SeaDAS-MUMM}
  		\end{overpic}
  	\end{minipage}
  	\begin{minipage}[c]{1.0\linewidth}
  		\centering
  		\vspace{0.5cm}
  		\begin{overpic}[trim=0 0 0 0,clip,height=1.2cm]{./Images/Collocated13262_ACOSWIR_MOB_SEA5x5_MUMM45_colorbar}
  		\put (28,16) {$R_{rs}(443nm) [1/sr]$}
  		\end{overpic}
  	\end{minipage}

  \caption{$R_{rs}$ 443.\label{fig:Rrs443} } 
\end{figure}
% %^^^^^^^^^^^^^^^^^^^  FIGURE ^^^^^^^^^^^^^^^^^^^^^^^^^^^^^^^^^^^^^^^^^^^^
% \begin{figure}[htbp!]
% 	\begin{minipage}[c]{0.48\linewidth}
%   		\centering
%   		\begin{overpic}[trim=0 150 40 150,clip,width=7.5cm]{./Images/Collocated13262_ACOSWIR_MOB_SEA5x5_MUMM45_Rrs_483_MOB}
%   		\put (5,6) {MOB-ELM}
%   		\end{overpic}
%   	\end{minipage}
%   	\hfill
% 	\begin{minipage}[c]{0.48\linewidth}
%   		\centering
%   		\begin{overpic}[trim=0 0 40 0,clip,width=7.5cm]{./Images/Collocated13262_ACOSWIR_MOB_SEA5x5_MUMM45_Rrs_483_ACO_R_R}
%   		\put (5,5) {ACOLITE-SWIR}
%   		\end{overpic}
%   	\end{minipage}

%   	\vspace{0.7cm}

% 	\begin{minipage}[c]{0.48\linewidth}
%   		\centering
%   		\begin{overpic}[trim=0 0 40 0,clip,width=7.5cm]{./Images/Collocated13262_ACOSWIR_MOB_SEA5x5_MUMM45_Rrs_482_SEA5x5_R}
%   		\put (5,5) {SEADAS-SWIR}
%   		\end{overpic}
%   	\end{minipage}
%   	\hfill
% 	\begin{minipage}[c]{0.48\linewidth}
%   		\centering
%   		\begin{overpic}[trim=0 150 40 150,clip,width=7.5cm]{./Images/Collocated13262_ACOSWIR_MOB_SEA5x5_MUMM45_Rrs_482_MUMM45}
%   		\put (5,5) {MUMM}
%   		\end{overpic}
%   	\end{minipage}
  	

%   	\begin{minipage}[c]{1.0\linewidth}
%   		\centering
%   		\vspace{0.5cm}
%   		\begin{overpic}[trim=0 0 0 0,clip,height=1.2cm]{./Images/Collocated13262_ACOSWIR_MOB_SEA5x5_MUMM45_colorbar}
%   		\put (28,16) {$R_{rs}(482nm) [1/sr]$}
%   		\end{overpic}
%   	\end{minipage}

%   \caption{$R_{rs}$ 483.\label{fig:Rrs482} } 
% \end{figure}
%^^^^^^^^^^^^^^^^^^^  FIGURE ^^^^^^^^^^^^^^^^^^^^^^^^^^^^^^^^^^^^^^^^^^^^
\begin{figure}[htbp!]
	\begin{minipage}[c]{0.48\linewidth}
  		\centering
  		\begin{overpic}[trim=0 155 40 150,clip,width=7.5cm]{./Images/Collocated13262_ACOSWIR_MOB_SEA5x5_MUMM45_Rrs_561_MOB}
  		\put (5,5) {MOB-ELM}
  		\end{overpic}
  	\end{minipage}
  	\hfill
	\begin{minipage}[c]{0.48\linewidth}
  		\centering
  		\begin{overpic}[trim=0 150 40 150,clip,width=7.5cm]{./Images/Collocated13262_ACOSWIR_MOB_SEA5x5_MUMM45_Rrs_561_ACO_R_R}
  		\put (5,5) {Acolite-SWIR}
  		\end{overpic}
  	\end{minipage}

  	\vspace{0.7cm}

	\begin{minipage}[c]{0.48\linewidth}
  		\centering
  		\begin{overpic}[trim=0 150 40 150,clip,width=7.5cm]{./Images/Collocated13262_ACOSWIR_MOB_SEA5x5_MUMM45_Rrs_561_SEA5x5_R}
  		\put (5,5) {SeaDAS-SWIR}
  		\end{overpic}
  	\end{minipage}
  	\hfill
	\begin{minipage}[c]{0.48\linewidth}
  		\centering
  		\begin{overpic}[trim=0 150 40 150,clip,width=7.5cm]{./Images/Collocated13262_ACOSWIR_MOB_SEA5x5_MUMM45_Rrs_561_MUMM45}
  		\put (5,5) {SeaDAS-MUMM}
  		\end{overpic}
  	\end{minipage}
  	

  	\begin{minipage}[c]{1.0\linewidth}
  		\centering
  		\vspace{0.5cm}
  		\begin{overpic}[trim=0 0 0 0,clip,height=1.2cm]{./Images/Collocated13262_ACOSWIR_MOB_SEA5x5_MUMM45_colorbar}
  		\put (28,16) {$R_{rs}(561nm) [1/sr]$}
  		\end{overpic}
  	\end{minipage}

  \caption{$R_{rs}$ 561.\label{fig:Rrs561} } 
\end{figure}
%^^^^^^^^^^^^^^^^^^^  FIGURE ^^^^^^^^^^^^^^^^^^^^^^^^^^^^^^^^^^^^^^^^^^^^
% \begin{figure}[htbp!]
% 	\begin{minipage}[c]{0.48\linewidth}
%   		\centering
%   		\begin{overpic}[trim=0 0 40 0,clip,width=7.5cm]{./Images/Collocated13262_ACOSWIR_MOB_SEA5x5_MUMM45_Rrs_655_MOB}
%   		\put (5,5) {MOB-ELM}
%   		\end{overpic}
%   	\end{minipage}
%   	\hfill
% 	\begin{minipage}[c]{0.48\linewidth}
%   		\centering
%   		\begin{overpic}[trim=0 0 40 0,clip,width=7.5cm]{./Images/Collocated13262_ACOSWIR_MOB_SEA5x5_MUMM45_Rrs_655_ACO_R_R}
%   		\put (5,5) {ACOLITE-SWIR}
%   		\end{overpic}
%   	\end{minipage}

%   	\vspace{0.7cm}

% 	\begin{minipage}[c]{0.48\linewidth}
%   		\centering
%   		\begin{overpic}[trim=0 0 40 0,clip,width=7.5cm]{./Images/Collocated13262_ACOSWIR_MOB_SEA5x5_MUMM45_Rrs_655_SEA5x5_R}
%   		\put (5,5) {SEADAS-SWIR}
%   		\end{overpic}
%   	\end{minipage}
%   	\hfill
% 	\begin{minipage}[c]{0.48\linewidth}
%   		\centering
%   		\begin{overpic}[trim=0 0 40 0,clip,width=7.5cm]{./Images/Collocated13262_ACOSWIR_MOB_SEA5x5_MUMM45_Rrs_655_MUMM45}
%   		\put (5,5) {MUMM}
%   		\end{overpic}
%   	\end{minipage}
  	

%   	\begin{minipage}[c]{1.0\linewidth}
%   		\centering
%   		\vspace{0.5cm}
%   		\begin{overpic}[trim=0 0 0 0,clip,height=1.2cm]{./Images/Collocated13262_ACOSWIR_MOB_SEA5x5_MUMM45_colorbar}
%   		\put (28,16) {$R_{rs}(655nm) [1/sr]$}
%   		\end{overpic}
%   	\end{minipage}

%   \caption{$R_{rs}$ 655.\label{fig:Rrs655} } 
% \end{figure}


%^^^^^^^^^^^^^^^^^^^  FIGURE ^^^^^^^^^^^^^^^^^^^^^^^^^^^^^^^^^^^^^^^^^^^^
\begin{figure}[htbp!]
  \begin{minipage}[c]{0.48\linewidth}
  		\centering
      \begin{overpic}[trim=0 280 0 0,clip,width=7.0cm]{./Images/2013262_ACOMOBSEAMUM_443_Acolite-SWIR_SeaDAS-MUMM.png}
      % \put (65,17) {\large A) $443nm$}
      \end{overpic}  
  \end{minipage}
  \hfill
  \begin{minipage}[d]{0.48\linewidth}
  	\centering
      \begin{overpic}[trim=0 280 0 0,clip,width=7.0cm]{./Images/2013262_ACOMOBSEAMUM_443_Acolite-SWIR_MoB-ELM.png}
      % \put (65,17) {\large B) $483nm$}
      \end{overpic}
  \end{minipage}

  \begin{minipage}[c]{0.48\linewidth}
  		\centering
      \begin{overpic}[trim=0 280 0 0,clip,width=7.0cm]{./Images/2013262_ACOMOBSEAMUM_443_Acolite-SWIR_SeaDAS-SWIR.png}
      % \put (65,17) {\large C) $561nm$}
      \end{overpic}  
  \end{minipage}
  \hfill
  \begin{minipage}[d]{0.48\linewidth}
  	\centering
      \begin{overpic}[trim=0 280 0 0,clip,width=7.0cm]{./Images/2013262_ACOMOBSEAMUM_443_SeaDAS-MUMM_MoB-ELM.png}
      % \put (65,17) {\large D) $655nm$}
      \end{overpic}
  \end{minipage}

  \begin{minipage}[c]{0.48\linewidth}
  		\centering
      \begin{overpic}[trim=0 280 0 0,clip,width=7.0cm]{./Images/2013262_ACOMOBSEAMUM_443_SeaDAS-SWIR_SeaDAS-MUMM.png}
      % \put (65,17) {\large C) $561nm$}
      \end{overpic}  
  \end{minipage}
  \hfill
  \begin{minipage}[d]{0.48\linewidth}
  	\centering
      \begin{overpic}[trim=0 280 0 0,clip,width=7.0cm]{./Images/2013262_ACOMOBSEAMUM_443_SeaDAS-SWIR_MoB-ELM.png}
      % \put (65,17) {\large D) $655nm$}
      \end{overpic}
  \end{minipage}

  \begin{minipage}[d]{1.0\linewidth}
  	\centering
      \begin{overpic}[trim=70 50 0 1450,clip,width=8.0cm]{./Images/2013262_ACOMOBSEAMUM_655_Acolite-SWIR_SeaDAS-MUMM.png}
      \end{overpic}
  \end{minipage}    

% 
  \caption{Scatter plots showing the comparison of remote-sensing reflectance ($R_{rs}$) at 443 nm, derived from the 09-29-2013 image over the Rochester Embayment (scene LC80160302013262LGN00) using the different methods. Colors denote pixel densities, the dashed black line is the 1:1 line, and the Reduced Major Axis (RMA) regression line is drawn in red. \label{fig:13262Rrs443} } 
\end{figure}

%^^^^^^^^^^^^^^^^^^^  FIGURE ^^^^^^^^^^^^^^^^^^^^^^^^^^^^^^^^^^^^^^^^^^^^
\begin{figure}[htbp!]
  \begin{minipage}[c]{0.48\linewidth}
  		\centering
      \begin{overpic}[trim=0 280 0 0,clip,width=7.0cm]{./Images/2013262_ACOMOBSEAMUM_561_Acolite-SWIR_SeaDAS-MUMM.png}
      % \put (65,17) {\large A) $443nm$}
      \end{overpic}  
  \end{minipage}
  \hfill
  \begin{minipage}[d]{0.48\linewidth}
  	\centering
      \begin{overpic}[trim=0 280 0 0,clip,width=7.0cm]{./Images/2013262_ACOMOBSEAMUM_561_Acolite-SWIR_MoB-ELM.png}
      % \put (65,17) {\large B) $483nm$}
      \end{overpic}
  \end{minipage}

  \begin{minipage}[c]{0.48\linewidth}
  		\centering
      \begin{overpic}[trim=0 280 0 0,clip,width=7.0cm]{./Images/2013262_ACOMOBSEAMUM_561_Acolite-SWIR_SeaDAS-SWIR.png}
      % \put (65,17) {\large C) $561nm$}
      \end{overpic}  
  \end{minipage}
  \hfill
  \begin{minipage}[d]{0.48\linewidth}
  	\centering
      \begin{overpic}[trim=0 280 0 0,clip,width=7.0cm]{./Images/2013262_ACOMOBSEAMUM_561_SeaDAS-MUMM_MoB-ELM.png}
      % \put (65,17) {\large D) $655nm$}
      \end{overpic}
  \end{minipage}

  \begin{minipage}[c]{0.48\linewidth}
  		\centering
      \begin{overpic}[trim=0 280 0 0,clip,width=7.0cm]{./Images/2013262_ACOMOBSEAMUM_561_SeaDAS-SWIR_SeaDAS-MUMM.png}
      % \put (65,17) {\large C) $561nm$}
      \end{overpic}  
  \end{minipage}
  \hfill
  \begin{minipage}[d]{0.48\linewidth}
  	\centering
      \begin{overpic}[trim=0 280 0 0,clip,width=7.0cm]{./Images/2013262_ACOMOBSEAMUM_561_SeaDAS-SWIR_MoB-ELM.png}
      % \put (65,17) {\large D) $655nm$}
      \end{overpic}
  \end{minipage}

  \begin{minipage}[d]{1.0\linewidth}
  	\centering
      \begin{overpic}[trim=70 50 0 1450,clip,width=8.0cm]{./Images/2013262_ACOMOBSEAMUM_655_Acolite-SWIR_SeaDAS-MUMM.png}
      \end{overpic}
  \end{minipage}    

% 
  \caption{Scatter plots showing the comparison of remote-sensing reflectance ($R_{rs}$) at 561 nm, derived from the 09-29-2013 image over the Rochester Embayment (scene LC80160302013262LGN00) using the the different methods. Colors denote pixel densities, the dashed black line is the 1:1 line, and the Reduced Major Axis (RMA) regression line is drawn in red. \label{fig:13262Rrs655} } 
\end{figure}

% %^^^^^^^^^^^^^^^^^^^  FIGURE ^^^^^^^^^^^^^^^^^^^^^^^^^^^^^^^^^^^^^^^^^^^^
% \begin{figure}[htbp!]
%   \begin{minipage}[c]{0.48\linewidth}
%   		\centering
%       \begin{overpic}[trim=250 310 250 0,clip,width=9cm]{./Images/2013262_ACOMOBSEAMUM_443_Acolite-SWIR_SeaDAS-MUMM.png}
%       \put (65,17) {\large A) $443nm$}
%       \end{overpic}  
%   \end{minipage}
%   \hfill
%   \begin{minipage}[d]{0.48\linewidth}
%   	\centering
%       \begin{overpic}[trim=250 310 250 0,clip,width=9cm]{./Images/2013262_ACOMOBSEAMUM_483_Acolite-SWIR_SeaDAS-MUMM.png}
%       \put (65,17) {\large B) $483nm$}
%       \end{overpic}
%   \end{minipage}

%   \begin{minipage}[c]{0.48\linewidth}
%   		\centering
%       \begin{overpic}[trim=250 310 250 0,clip,width=9cm]{./Images/2013262_ACOMOBSEAMUM_561_Acolite-SWIR_SeaDAS-MUMM.png}
%       \put (65,17) {\large C) $561nm$}
%       \end{overpic}  
%   \end{minipage}
%   \hfill
%   \begin{minipage}[d]{0.48\linewidth}
%   	\centering
%       \begin{overpic}[trim=250 310 250 0,clip,width=9cm]{./Images/2013262_ACOMOBSEAMUM_655_Acolite-SWIR_SeaDAS-MUMM.png}
%       \put (65,17) {\large D) $655nm$}
%       \end{overpic}
%   \end{minipage}

%   \begin{minipage}[d]{1.0\linewidth}
%   	\centering
%       \begin{overpic}[trim=70 50 0 1450,clip,width=9cm]{./Images/2013262_ACOMOBSEAMUM_655_Acolite-SWIR_SeaDAS-MUMM.png}
%       \end{overpic}
%   \end{minipage}    

% % 
%   \caption{Scatter plots showing the comparison of remote-sensing reflectance ($R_{rs}$) at 443, 483, 561, 655 and 865 nm, derived from the 09-29-2013 image over the Rochester Embayment (scene LC80160302013262LGN00) using the ACOLITE tool (x) and the MUMM algorithm (y). Colors denote pixel densities, the dashed black line is the 1:1 line, and the Reduced Major Axis (RMA) regression line is drawn in red. \label{fig:13262RrsAcolite-SWIR_SeaDAS-MUMM} } 
% \end{figure}
% %^^^^^^^^^^^^^^^^^^^  FIGURE ^^^^^^^^^^^^^^^^^^^^^^^^^^^^^^^^^^^^^^^^^^^^
% \begin{figure}[htbp!]
%   \begin{minipage}[c]{0.48\linewidth}
%   		\centering
%       \begin{overpic}[trim=250 310 250 0,clip,width=9cm]{./Images/2013262_ACOMOBSEAMUM_443_Acolite-SWIR_MoB-ELM.png}
%       \put (65,17) {\large A) $443nm$}
%       \end{overpic}  
%   \end{minipage}
%   \hfill
%   \begin{minipage}[d]{0.48\linewidth}
%   	\centering
%       \begin{overpic}[trim=250 310 250 0,clip,width=9cm]{./Images/2013262_ACOMOBSEAMUM_483_Acolite-SWIR_MoB-ELM.png}
%       \put (65,17) {\large B) $483nm$}
%       \end{overpic}
%   \end{minipage}

%   \begin{minipage}[c]{0.48\linewidth}
%   		\centering
%       \begin{overpic}[trim=250 310 250 0,clip,width=9cm]{./Images/2013262_ACOMOBSEAMUM_561_Acolite-SWIR_MoB-ELM.png}
%       \put (65,17) {\large C) $561nm$}
%       \end{overpic}  
%   \end{minipage}
%   \hfill
%   \begin{minipage}[d]{0.48\linewidth}
%   	\centering
%       \begin{overpic}[trim=250 310 250 0,clip,width=9cm]{./Images/2013262_ACOMOBSEAMUM_655_Acolite-SWIR_MoB-ELM.png}
%       \put (65,17) {\large D) $655nm$}
%       \end{overpic}
%   \end{minipage}

%   \begin{minipage}[d]{1.0\linewidth}
%   	\centering
%       \begin{overpic}[trim=70 50 0 1450,clip,width=9cm]{./Images/2013262_ACOMOBSEAMUM_655_Acolite-SWIR_MoB-ELM.png}
%       \end{overpic}
%   \end{minipage}    

% % 
%   \caption{Scatter plots showing the comparison of remote-sensing reflectance ($R_{rs}$) at 443, 483, 561, 655 and 865 nm, derived from the 09-29-2013 image over the Rochester Embayment (scene LC80160302013262LGN00) using the ACOLITE tool (x) and the MoB-ELM algorithm (y). Colors denote pixel densities, the dashed black line is the 1:1 line, and the Reduced Major Axis (RMA) regression line is drawn in red. \label{fig:13262RrsAcolite-SWIR_MoB-ELM} } 
% \end{figure}
% %^^^^^^^^^^^^^^^^^^^  FIGURE ^^^^^^^^^^^^^^^^^^^^^^^^^^^^^^^^^^^^^^^^^^^^
% \begin{figure}[htbp!]
%   \begin{minipage}[c]{0.48\linewidth}
%   		\centering
%       \begin{overpic}[trim=250 310 250 0,clip,width=9cm]{./Images/2013262_ACOMOBSEAMUM_443_Acolite-SWIR_SeaDAS-SWIR.png}
%       \put (65,17) {\large A) $443nm$}
%       \end{overpic}  
%   \end{minipage}
%   \hfill
%   \begin{minipage}[d]{0.48\linewidth}
%   	\centering
%       \begin{overpic}[trim=250 310 250 0,clip,width=9cm]{./Images/2013262_ACOMOBSEAMUM_483_Acolite-SWIR_SeaDAS-SWIR.png}
%       \put (65,17) {\large B) $483nm$}
%       \end{overpic}
%   \end{minipage}

%   \begin{minipage}[c]{0.48\linewidth}
%   		\centering
%       \begin{overpic}[trim=250 310 250 0,clip,width=9cm]{./Images/2013262_ACOMOBSEAMUM_561_Acolite-SWIR_SeaDAS-SWIR.png}
%       \put (65,17) {\large C) $561nm$}
%       \end{overpic}  
%   \end{minipage}
%   \hfill
%   \begin{minipage}[d]{0.48\linewidth}
%   	\centering
%       \begin{overpic}[trim=250 310 250 0,clip,width=9cm]{./Images/2013262_ACOMOBSEAMUM_655_Acolite-SWIR_SeaDAS-SWIR.png}
%       \put (65,17) {\large D) $655nm$}
%       \end{overpic}
%   \end{minipage}

%   \begin{minipage}[d]{1.0\linewidth}
%   	\centering
%       \begin{overpic}[trim=70 50 0 1450,clip,width=9cm]{./Images/2013262_ACOMOBSEAMUM_655_Acolite-SWIR_SeaDAS-SWIR.png}
%       \end{overpic}
%   \end{minipage}    

% % 
%   \caption{Scatter plots showing the comparison of remote-sensing reflectance ($R_{rs}$) at 443, 483, 561, 655 and 865 nm, derived from the 09-29-2013 image over the Rochester Embayment (scene LC80160302013262LGN00) using the ACOLITE tool (x) and the SeaDAS algorithm (y). Colors denote pixel densities, the dashed black line is the 1:1 line, and the Reduced Major Axis (RMA) regression line is drawn in red. \label{fig:13262RrsAcolite-SWIR_SeaDAS} } 
% \end{figure}
% %^^^^^^^^^^^^^^^^^^^  FIGURE ^^^^^^^^^^^^^^^^^^^^^^^^^^^^^^^^^^^^^^^^^^^^
% \begin{figure}[htbp!]
%   \begin{minipage}[c]{0.48\linewidth}
%   		\centering
%       \begin{overpic}[trim=250 310 250 0,clip,width=9cm]{./Images/2013262_ACOMOBSEAMUM_443_SeaDAS-MUMM_MoB-ELM.png}
%       \put (65,17) {\large A) $443nm$}
%       \end{overpic}  
%   \end{minipage}
%   \hfill
%   \begin{minipage}[d]{0.48\linewidth}
%   	\centering
%       \begin{overpic}[trim=250 310 250 0,clip,width=9cm]{./Images/2013262_ACOMOBSEAMUM_483_SeaDAS-MUMM_MoB-ELM.png}
%       \put (65,17) {\large B) $483nm$}
%       \end{overpic}
%   \end{minipage}

%   \begin{minipage}[c]{0.48\linewidth}
%   		\centering
%       \begin{overpic}[trim=250 310 250 0,clip,width=9cm]{./Images/2013262_ACOMOBSEAMUM_561_SeaDAS-MUMM_MoB-ELM.png}
%       \put (65,17) {\large C) $561nm$}
%       \end{overpic}  
%   \end{minipage}
%   \hfill
%   \begin{minipage}[d]{0.48\linewidth}
%   	\centering
%       \begin{overpic}[trim=250 310 250 0,clip,width=9cm]{./Images/2013262_ACOMOBSEAMUM_655_SeaDAS-MUMM_MoB-ELM.png}
%       \put (65,17) {\large D) $655nm$}
%       \end{overpic}
%   \end{minipage}

%   \begin{minipage}[d]{1.0\linewidth}
%   	\centering
%       \begin{overpic}[trim=70 50 0 1450,clip,width=9cm]{./Images/2013262_ACOMOBSEAMUM_655_SeaDAS-MUMM_MoB-ELM.png}
%       \end{overpic}
%   \end{minipage}    

% % 
%   \caption{Scatter plots showing the comparison of remote-sensing reflectance ($R_{rs}$) at 443, 483, 561, 655 and 865 nm, derived from the 09-29-2013 image over the Rochester Embayment (scene LC80160302013262LGN00) using the MUMM algorithm in SeaDAS (x) and the MoB-ELM algorithm (y). Colors denote pixel densities, the dashed black line is the 1:1 line, and the Reduced Major Axis (RMA) regression line is drawn in red. \label{fig:13262RrsSeaDAS-MUMM_MoB-ELM} } 
% \end{figure}
% %^^^^^^^^^^^^^^^^^^^  FIGURE ^^^^^^^^^^^^^^^^^^^^^^^^^^^^^^^^^^^^^^^^^^^^
% \begin{figure}[htbp!]
%   \begin{minipage}[c]{0.48\linewidth}
%   		\centering
%       \begin{overpic}[trim=250 310 250 0,clip,width=9cm]{./Images/2013262_ACOMOBSEAMUM_443_SeaDAS-SWIR_SeaDAS-MUMM.png}
%       \put (65,17) {\large A) $443nm$}
%       \end{overpic}  
%   \end{minipage}
%   \hfill
%   \begin{minipage}[d]{0.48\linewidth}
%   	\centering
%       \begin{overpic}[trim=250 310 250 0,clip,width=9cm]{./Images/2013262_ACOMOBSEAMUM_483_SeaDAS-SWIR_SeaDAS-MUMM.png}
%       \put (65,17) {\large B) $483nm$}
%       \end{overpic}
%   \end{minipage}

%   \begin{minipage}[c]{0.48\linewidth}
%   		\centering
%       \begin{overpic}[trim=250 310 250 0,clip,width=9cm]{./Images/2013262_ACOMOBSEAMUM_561_SeaDAS-SWIR_SeaDAS-MUMM.png}
%       \put (65,17) {\large C) $561nm$}
%       \end{overpic}  
%   \end{minipage}
%   \hfill
%   \begin{minipage}[d]{0.48\linewidth}
%   	\centering
%       \begin{overpic}[trim=250 310 250 0,clip,width=9cm]{./Images/2013262_ACOMOBSEAMUM_655_SeaDAS-SWIR_SeaDAS-MUMM.png}
%       \put (65,17) {\large D) $655nm$}
%       \end{overpic}
%   \end{minipage}

%   \begin{minipage}[d]{1.0\linewidth}
%   	\centering
%       \begin{overpic}[trim=70 50 0 1450,clip,width=9cm]{./Images/2013262_ACOMOBSEAMUM_655_SeaDAS-SWIR_SeaDAS-MUMM.png}
%       \end{overpic}
%   \end{minipage}    

% % 
%   \caption{Scatter plots showing the comparison of remote-sensing reflectance ($R_{rs}$) at 443, 483, 561, 655 and 865 nm, derived from the 09-29-2013 image over the Rochester Embayment (scene LC80160302013262LGN00) using the SeaDAS tool (x) and the MUMM algorithm  in SeaDAS (y). Colors denote pixel densities, the dashed black line is the 1:1 line, and the Reduced Major Axis (RMA) regression line is drawn in red. \label{fig:13262RrsSeaDAS_SeaDAS-MUMM} } 
% \end{figure}
% %^^^^^^^^^^^^^^^^^^^  FIGURE ^^^^^^^^^^^^^^^^^^^^^^^^^^^^^^^^^^^^^^^^^^^^
% \begin{figure}[htbp!]
%   \begin{minipage}[c]{0.48\linewidth}
%   		\centering
%       \begin{overpic}[trim=250 310 250 0,clip,width=9cm]{./Images/2013262_ACOMOBSEAMUM_443_SeaDAS-SWIR_MoB-ELM.png}
%       \put (65,17) {\large A) $443nm$}
%       \end{overpic}  
%   \end{minipage}
%   \hfill
%   \begin{minipage}[d]{0.48\linewidth}
%   	\centering
%       \begin{overpic}[trim=250 310 250 0,clip,width=9cm]{./Images/2013262_ACOMOBSEAMUM_483_SeaDAS-SWIR_MoB-ELM.png}
%       \put (65,17) {\large B) $483nm$}
%       \end{overpic}
%   \end{minipage}

%   \begin{minipage}[c]{0.48\linewidth}
%   		\centering
%       \begin{overpic}[trim=250 310 250 0,clip,width=9cm]{./Images/2013262_ACOMOBSEAMUM_561_SeaDAS-SWIR_MoB-ELM.png}
%       \put (65,17) {\large C) $561nm$}
%       \end{overpic}  
%   \end{minipage}
%   \hfill
%   \begin{minipage}[d]{0.48\linewidth}
%   	\centering
%       \begin{overpic}[trim=250 310 250 0,clip,width=9cm]{./Images/2013262_ACOMOBSEAMUM_655_SeaDAS-SWIR_MoB-ELM.png}
%       \put (65,17) {\large D) $655nm$}
%       \end{overpic}
%   \end{minipage}

%   \begin{minipage}[d]{1.0\linewidth}
%   	\centering
%       \begin{overpic}[trim=70 50 0 1450,clip,width=9cm]{./Images/2013262_ACOMOBSEAMUM_655_SeaDAS-SWIR_MoB-ELM.png}
%       \end{overpic}
%   \end{minipage}    

% % 
%   \caption{Scatter plots showing the comparison of remote-sensing reflectance ($R_{rs}$) at 443, 483, 561, 655 and 865 nm, derived from the 09-29-2013 image over the Rochester Embayment (scene LC80160302013262LGN00) using the SeaDAS tool (x) and the MoB-ELM algorithm (y). Colors denote pixel densities, the dashed black line is the 1:1 line, and the Reduced Major Axis (RMA) regression line is drawn in red. \label{fig:13262RrsSeaDAS_MoB-ELM} } 
% \end{figure}
%^^^^^^^^^^^^^^^^^^^  FIGURE ^^^^^^^^^^^^^^^^^^^^^^^^^^^^^^^^^^^^^^^^^^^^
\begin{figure}[htbp!]
	\begin{minipage}[c]{0.48\linewidth}
  		\centering
  		\begin{overpic}[trim=0 0 40 0,clip,width=7.5cm]{./Images/Collocated13262_ACOSWIR_MOB_SEA5x5_MUMM45_chlor_MOB_D_R_R}
  		\put (5,5) {MOB-ELM}
  		\end{overpic}
  	\end{minipage}
  	\hfill
	\begin{minipage}[c]{0.48\linewidth}
  		\centering
  		\begin{overpic}[trim=0 0 40 0,clip,width=7.5cm]{./Images/LC80160302013262LGN00_L2_SWIR_FranzAve_chlor_a_ACO_OC3def}
  		\put (5,5) {Acolite-SWIR}
  		\end{overpic}
  	\end{minipage}

  	\vspace{0.7cm}

	\begin{minipage}[c]{0.48\linewidth}
  		\centering
  		\begin{overpic}[trim=0 0 40 0,clip,width=7.5cm]{./Images/Collocated13262_ACOSWIR_MOB_SEA5x5_MUMM45_chlor_a_SEA5x5_R}
  		\put (5,5) {SeaDAS-SWIR}
  		\end{overpic}
  	\end{minipage}
  	\hfill
	\begin{minipage}[c]{0.48\linewidth}
  		\centering
  		\begin{overpic}[trim=0 0 40 0,clip,width=7.5cm]{./Images/Collocated13262_ACOSWIR_MOB_SEA5x5_MUMM45_chlor_a_MUMM45}
  		\put (5,5) {SeaDAS-MUMM}
  		\end{overpic}
  	\end{minipage}
  	

  	\begin{minipage}[c]{1.0\linewidth}
  		\centering
  		\vspace{0.5cm}
  		\begin{overpic}[trim=0 0 0 0,clip,height=1.2cm]{./Images/Collocated13262_ACOSWIR_MOB_SEA5x5_MUMM45_colorbar_CHL_0_100}
  		\put (35,16) {$C_a [mg/m^3]$}
  		\end{overpic}
  	\end{minipage}

  \caption{$C_a.$ \label{fig:chlor_a} } 
\end{figure}
%^^^^^^^^^^^^^^^^^^^  FIGURE ^^^^^^^^^^^^^^^^^^^^^^^^^^^^^^^^^^^^^^^^^^^^
\begin{figure}[htbp!]
  \begin{minipage}[c]{0.48\linewidth}
  		\centering
      \begin{overpic}[trim=0 250 0 0,clip,width=7cm]{./Images/2013262_ACOMOBSEAMUM_C_a_Acolite-SWIR_SeaDAS-MUMM.png}
      \end{overpic}  
  \end{minipage}
  \hfill
  \begin{minipage}[d]{0.48\linewidth}
  	\centering
      \begin{overpic}[trim=0 250 0 0,clip,width=7cm]{./Images/2013262_ACOMOBSEAMUM_C_a_Acolite-SWIR_MoB-ELM.png}
      \end{overpic}
  \end{minipage}

  \begin{minipage}[c]{0.48\linewidth}
  		\centering
      \begin{overpic}[trim=0 250 0 0,clip,width=7cm]{./Images/2013262_ACOMOBSEAMUM_C_a_Acolite-SWIR_SeaDAS-SWIR.png}
      \end{overpic}  
  \end{minipage}
  \hfill
  \begin{minipage}[d]{0.48\linewidth}
  	\centering
      \begin{overpic}[trim=0 250 0 0,clip,width=7cm]{./Images/2013262_ACOMOBSEAMUM_C_a_SeaDAS-MUMM_MoB-ELM.png}
      \end{overpic}
  \end{minipage}

  \begin{minipage}[c]{0.48\linewidth}
  		\centering
      \begin{overpic}[trim=0 250 0 0,clip,width=7cm]{./Images/2013262_ACOMOBSEAMUM_C_a_SeaDAS-SWIR_SeaDAS-MUMM.png}
      \end{overpic}  
  \end{minipage}
  \hfill
  \begin{minipage}[d]{0.48\linewidth}
  	\centering
      \begin{overpic}[trim=0 250 0 0,clip,width=7cm]{./Images/2013262_ACOMOBSEAMUM_C_a_SeaDAS-SWIR_MoB-ELM.png}
      \end{overpic}
  \end{minipage}

  \begin{minipage}[d]{1.0\linewidth}
  	\centering
      \begin{overpic}[trim=0 0 0 1500,clip,width=7cm]{./Images/2013262_ACOMOBSEAMUM_C_a_SeaDAS-SWIR_MoB-ELM.png}
      \end{overpic}
  \end{minipage}    

\vspace{.5cm}
  \caption{$C_a$ comparison among all the four method analyzed in this work. \label{fig:13262Chlor} } 
\end{figure}
%^^^^^^^^^^^^^^^^^^^  FIGURE ^^^^^^^^^^^^^^^^^^^^^^^^^^^^^^^^^^^^^^^^^^^^
\begin{figure}[htbp!]
  \begin{minipage}[c]{0.3\linewidth}
  		\centering
      \begin{overpic}[trim=0 0 0 0,clip,width=5.0cm]{./Images/RrsCompONTNS.png}
      \put (20,60) {G) ONTNS} 
      \end{overpic}  
  \end{minipage}
  \hfill
  \begin{minipage}[d]{0.3\linewidth}
  	\centering
      \begin{overpic}[trim=0 0 0 0,clip,width=5.0cm]{./Images/RrsCompONTOS.png}
      \put (20,14) {H) ONTOS}  	 	
      \end{overpic}
  \end{minipage}
  \hfill
  \begin{minipage}[d]{0.3\linewidth}
  	\centering
      \begin{overpic}[trim=0 0 0 0,clip,width=5.0cm]{./Images/RrsCompONTEX.png}
      \put (20,14) {F) ONTEX}  	

      \end{overpic}
  \end{minipage}

  \begin{minipage}[c]{0.3\linewidth}
  		\centering
      \begin{overpic}[trim=0 0 0 0,clip,width=5.0cm]{./Images/RrsCompRVRPL.png}
      \put (20,14) {I) RVRPL}  		

      \end{overpic}  
  \end{minipage}
  \hfill
  \begin{minipage}[d]{0.3\linewidth}
  	\centering
      \begin{overpic}[trim=0 0 0 0,clip,width=5.0cm]{./Images/RrsCompLONGN.png}
      \put (20,60) {D) LONGN}  	

      \end{overpic}
  \end{minipage}
  \hfill
  \begin{minipage}[d]{0.3\linewidth}
  	\centering
      \begin{overpic}[trim=0 0 0 0,clip,width=5.0cm]{./Images/RrsCompLONGS.png}
      \put (20,60) {E) LONGS}
      \end{overpic}
  \end{minipage}

   \begin{minipage}[c]{0.3\linewidth}
  		\centering
      \begin{overpic}[trim=0 0 0 0,clip,width=5.0cm]{./Images/RrsCompCRANB.png}
      \put (45,14) {C) CRANB}  		

      \end{overpic}  
  \end{minipage}
  \hfill
  \begin{minipage}[d]{0.3\linewidth}
  	\centering
      \begin{overpic}[trim=0 0 0 0,clip,width=5.0cm]{./Images/RrsCompBRADIN.png}
      \put (45,14) {A) BRADIN}
      \end{overpic}
  \end{minipage}
  \hfill
  \begin{minipage}[d]{0.3\linewidth}
  	\centering
      \begin{overpic}[trim=0 0 0 0,clip,width=5.0cm]{./Images/RrsCompBRADONT.png}
      \put (45,14) {B) BRADONT}

      \end{overpic}
  \end{minipage}    

% 
  \caption{Comparison of $R_{rs}$ for the sites on the 90-19-2013 collection. \label{fig:13262RrsComp}} 
\end{figure}
%^^^^^^^^^^^^^^^^^^^  FIGURE ^^^^^^^^^^^^^^^^^^^^^^^^^^^^^^^^^^^^^^^^^^^^
\begin{figure}[htbp!]
  \begin{minipage}[c]{1.0\linewidth}
		\centering
     	\begin{overpic}[trim=0 0 0 0,clip,width=12cm]{./Images/CHLmeavsret.png}
    	\put(17,60){\includegraphics[trim=10 80 0 0,clip,width=4cm]{./Images/CHLmeavsretZOOM.png}}
      \end{overpic}  
  \end{minipage}
  \caption{Comparison retrieved versus measured $R_{rs}$ for the sites on the 90-19-2013 collection. \label{fig:13262RrsMeaVSRet}} 
\end{figure}
% -----------------------------------------------------------
\subsection{Comparison of Chlorophyll-{\it a}}
%-------------
%^^^^^^^^^^^^^^^^^^^  FIGURE ^^^^^^^^^^^^^^^^^^^^^^^^^^^^^^^^^^^^^^^^^^^^
\begin{figure}[htbp!]
  \centering
  \includegraphics[height=8.0cm]{./Images/13262_NRME_CHL.png}
  \caption{NRMSE for $C_a$.\label{fig:NRMSE130919} } 
\end{figure}

%%%%%%%%%%%%%%%%%%%%%%%%%%%%%%%%%%%%%%%%%%%%%%%%%%%%%%%%%%%%%
\acknowledgments     %>>>> equivalent to \section*{ACKNOWLEDGMENTS}       
The authors would like to extend a special thanks to the United States Geological Survey (USGS) for its sponsorship that has made this effort possible.
%%%%%%%%%%%%%%%%%%%%%%%%%%%%%%%%%%%%%%%%%%%%%%%%%%%%%%%%%%%%%
%%%%% References %%%%%

\bibliography{/Users/javier/Desktop/Javier/PHD_RIT/Latex/javier_bib}   
\bibliographystyle{spiebib}   %>>>> makes bibtex use spiebib.bst


\end{document} 
